\documentclass[notitlepage,a4paper,11pt]{article}

\usepackage[no-solutions]{optional}
\usepackage{../common/preamble}
\renewcommand\sectionbreak{}
\usepackage{a4wide}
\author{Lecturers PT and PD}
\date{\today}
\setlength{\parindent}{0pt}

\setcounter{theorem}{46}
\begin{exercise} BH.8.47.
\begin{hint}
Realize that when $\max\{X, Y\} \leq z$, then $X\leq z$ \emph{and} $Y\leq z$. Use this to find the CDF of $M_{n}$. Next, if $M_{n+1}\leq y$, can it happen that $M_{n}> x$ for some $x>y$?
\end{hint}
\begin{solution}
With the hint, $M_{n}\leq x \implies X_{i}\leq x$ for all $i=1,\ldots, n$. Thus, $\P{M_n\leq x} = (F(x))^{n}$.

Now we go for the joint distribution of $M_{n}$ and $M_{n+1}$. If $y\leq x$, then from the hint,
\begin{align*}
\P{M_n\leq x, M_{n+1}\leq y} = \P{M_{n+1}\leq y} = (F(y))^{n+1}.
\end{align*}
When $y>x$, then we know that $M_{n}\leq x$ but $X_{n+1}\leq y$, hence,
\begin{align*}
\P{M_n\leq x, M_{n+1}\leq y} = \P{M_{n}\leq x, X_{n+1}\leq y} = (F(x))^{n} F(y).
\end{align*}
Where do we use the fact that the $X_i$ are iid?
\end{solution}
\end{exercise}
\input{trailer.tex}
