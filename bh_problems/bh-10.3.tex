\documentclass[notitlepage,a4paper,11pt]{article}

\usepackage[no-solutions]{optional}
\usepackage{../common/preamble}
\renewcommand\sectionbreak{}
\usepackage{a4wide}
\author{Lecturers PT and PD}
\date{\today}
\setlength{\parindent}{0pt}

\setcounter{theorem}{2}
\begin{exercise} BH.10.3.
Remark: This is just a funny exercise, but I wonder whether it has a practical value.
\begin{hint}
First check the assumption that $Y\neq a X$, for some $a>0$; why is it there?
Then, take a suitable $g$ in Jensen's inequality.
Bigger hint: $g(x)=1/x$.

In the solution guide, the authors do not explain the $>$, while in Jensen's inequality there is a $\leq$.
To see why the $>$ is allowed here, rethink the assumption in the exercise, and reread Theorem 10.1.5 carefully.

Finally, at what $p$ is $p(1-p)$ maximal?
\end{hint}
\begin{solution}
Check the hint.

With the Cauchy-Schwartz inequality we will not get what we want. To see why, take $W=\sqrt{X/Y}$ and $Z=\sqrt{Y/X}$, which is ok because $X,Y>0$.  Then, with the Cauchy-Schwartz inequality, $1 = (\E{WY})^{2} \leq \E{W^2}\E{Z^{2}} = \E{X/Y}\E{Y/X}$. But this gives only a $\leq$, not the $<$ which we have to prove.

See the solutions provided by BH in their pdf with solutions.
\end{solution}

\end{exercise}
\input{trailer.tex}
