\documentclass[notitlepage,a4paper,11pt]{article}

\usepackage[no-solutions]{optional}
\usepackage{../common/preamble}
\renewcommand\sectionbreak{}
\usepackage{a4wide}
\author{Lecturers PT and PD}
\date{\today}
\setlength{\parindent}{0pt}

\setcounter{theorem}{53}
\begin{exercise} BH.8.54.
%Remark: We tackle this also with simulation in an assignment.

Remark: I find it easier to consider $Y=pX$, rather than $pX/q$. Note that since $q=1-p\to 1$ as $p\to 0$, the factor $1/q$ is immaterial for the final result.
Besides the solution provided by BH in their pdf with selected solutions, read our solution too, as we develop some nice ideas in passing.

\begin{hint}
Use BH.4.3.9. Then, start with a geometic rv, then extend to a negative binomial rv.
\end{hint}
\begin{solution}
\begin{align*}
M_{Y}(s) =\E{\exp{s Y}} = p \sum_{i=0}^{\infty} e^{s p i} q^{i} = p/(1-qe^{sp}).
\end{align*}
Now, use that $e^{s p)} \approx 1 + s p$ for $p\ll 1$. (This is easier than using l' Hopital's rule as BH do in their solution). Hence, the denominator becomes $\approx 1-(1-p)(1+sp) = p(1-s) - sp^{2} \approx p(1-s)$ when $p\ll 1$ Hence,
\begin{align*}
M_{Y}(s) \approx p/(p(1-s) =  1/(1-s).
\end{align*}
In the limit $p\to 0$ the LHS converges to the RHS, which is the MGF of an exponential rv. For the rest, follow the solution of BH.

Here is another line of attack. Let us first use probability theory to find out what is $\sum_{i=0}^{\infty} q^{1}$ for some $|q|<1$. Take $X\sim\Geo{p}$, so that $X$ corresponds to the number of failures (tails say) until we see a success (heads say). So, $X$ corresponds to the number of tails until we see a heads. Now if we keep on throwing, then we know that eventually a heads will appear. Therefore $p + pq +pq^2 + \cdots = 1$, that, is $p\sum_i^{\infty} q^{i}=1$. But this implies that $\sum_{i=0}^{\infty}q^i = 1/ p = 1/(1-q)$.

By similar reasoning, if we keep on throwing the coin until we see $r$ heads then we know that $p^r \sum_{i=0}^{\infty} {r+i-1 \choose r} q^{i} = 1$.  Therefore,
\begin{align*}
\sum_{i=0}^{\infty} {r+i-1 \choose r} q^{i} =  \frac{1}{p^{2}}= \frac{1}{(1-q)^{r}}.
\end{align*}
With this insight, for $X\sim\NBin{p,n}$
\begin{align*}
  M_X(s) &= p^r \sum_{i=0}^{\infty} {r+i-1 \choose r} q^{i} e^{si}
 = p^r \sum_{i=0}^{\infty} {r+i-1 \choose r} (e^{s}q)^{i} \\
  &= \frac{p^r}{(1-qe^{s})^{r}} \approx \left(\frac{p}{p(1-s)}\right)^{r},
\end{align*}
where we use again Taylor's expansion for $p\ll 1$.x
\end{solution}
\end{exercise}
\input{trailer.tex}
