\documentclass[notitlepage,a4paper,11pt]{article}

\usepackage[no-solutions]{optional}
\usepackage{../common/preamble}
\renewcommand\sectionbreak{}
\usepackage{a4wide}
\author{Lecturers PT and PD}
\date{\today}
\setlength{\parindent}{0pt}

\setcounter{theorem}{57}
\begin{exercise} BH.9.58.
Remark: In part c.\/ the prior is the uniform distribution. What would happen if you would take the prior of part b, i.e., $a$ out of $j$ wins?
\begin{hint}
a. Recall that the uniform distribution on $[0,1]$ is $\Beta{a,b}$ with $a=b=1$. I prefer to write $S_n=\sum_{j=1}^n X_j$. First compute $\E{S_n\given p}$. Then compute $\E{\E{S_n| p}}$. Note that the outer expectation is an integral with respect to $p$ and the density of $\Beta{1,1}$.

For the variance, use Eve's law.

b. Use Beta-Binomial conjugacy. Or use the insights of BH.9.56 and BH.9.57.

c. Bayes Billiards.
\end{hint}

\begin{solution}
\begin{align*}
\E{S_n|p} &= np \\
\E{p} &= \frac 1{\beta(a,b)}\int_0^1pp^{a-1}q^{b-1}\d p = \frac{\beta(a+1,b)}{\beta(a,b)} = \frac a{a+b} = 1/2\\
\E{\E{S_n|p}} &= n\E{p} = n/2. \\
\V{S_n|p}  &= np q\\
\E{\V{S_n|p}}  &= n\E{p q} = n \E p - n \E{p^{2}} = n/2 - n\E{p^{2}}\\
\E{p^2} &=  \frac 1{\beta(a,b)}\int_0^1p^{2}p^{a-1}q^{b-1}\d p =
\frac{\beta(a+2,b)}{\beta(a,b)} = \frac{a(a+1)}{(a+b)(a+b+1)} = \frac 2{2\cdot 3} = 1/3\\
\V{\E{S_{n}|p}} &= \V{n p} = n^{2} \V p = n^{2}/12.
\end{align*}
The rest of Eve's law is now trivial.

b. We start with a $\Beta{1,1}$ prior on $p$. After the first win, the prior gets updated to $\Beta{1+1, 1}$, after a loss to $\Beta{1, 1+1}$. Reasoning like this, after $a$ wins and $j-a$ losses, the distribution for a win becomes $\Beta{1+a, a+ j-1}$. Therefore, by using the hint in the book,  $\E{p | S_j=a } = (a+1)/(j+2)$.

c. When somebody doesn't give me any information about what team can win, then any outcome must be equally likely. (What else can it be?)
This is also my way to understand the expression in BH.8.3.2. Hence, $\P{X=k} = 1/(n+1)$. Observe that we use the prior $p\sim\Beta{1,1}$.

When the prior is $\Beta{a, j-a}$, we should get the negative hypergeometric distribution, see the remark  in BH.8.3.3.

d. Shanille scores the first and missed the second. Hence, there are 98 shots left, out which she has to score $49$.  Thus, we ask for $\P{S_{98}=49|p}$, where $p\sim \Beta{a=1,b=1}$ is the prior since she hit $a=1$  out of $a+b=2$ shots. This places us in the situation of part c above, with $n=98$. Hence, $\P{S_{98}=49|p} = 1/99$.
\end{solution}
\end{exercise}
\input{trailer.tex}
