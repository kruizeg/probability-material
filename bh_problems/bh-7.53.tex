\documentclass[notitlepage,a4paper,11pt]{article}

\usepackage[no-solutions]{optional}
\usepackage{../common/preamble}
\renewcommand\sectionbreak{}
\usepackage{a4wide}
\author{Lecturers PT and PD}
\date{\today}
\setlength{\parindent}{0pt}

\setcounter{theorem}{52}
\begin{exercise}BH.7.53.
Remark: We simulate this in one of the assignments.
The ideas of this exercise find much use in finance, physics, and actuarial sciences.
In particular, the expected time it takes the drunken person---It's not only guys that sometimes consume too much alcohol---to hit some boundary is interesting. The notation of the book is a bit clumsy. Here is better notation.
Let $X_i$ be the movement along the \(x\)-axis at step $i$, and $Y_i$ along the $y$-axis.
Then $S_n=\sum_{i=1}^n X_i$ and $T_n=\sum_{j=1}^n Y_{j}$, and $R_n^2= S_n^2+T_n^2$.
\begin{hint}
Use the hint of the book and independence to see that $\E{S_{n}^2 T_n^2} = \E{S_n^{2}} \E{T_n^{2}}$.
Then try to simplify.

b. It is immediate that $\E{S_n} = 0$.
Hence, focus on $\E{S_n T_n}$. Expand  the sums of $\E{S_n T_n}$, and consider the individual terms $\E{X_i Y_j}$. When $i\neq j$, are $X_i$ and $Y_{j}$  independent? What if  $i=j$?

c. It is clear that $R_n^2=S_n^2+T_{n}^2$. Now use linearity to split $\E{R^2_n}$. Finally, realize that $\E{S_n}=0$, hence $\E{S_n^2} = \V{S_n}$. But then we can use the formula of the variance of a sum to split it up into a sum of variances plus covariances.
\end{hint}

\begin{solution}
a. In my notation, $X_i=0 \implies Y_i\neq 0$ and $X_i\neq 0 \implies Y_i=0$. The reason is that in step $i$, the drunkard makes a step left or right OR up or down. However, s/he cannot move to the right and up at the same time.

Here is an argument based on recursion. (By now I hope you see that I like this method in particular).
\begin{align*}
\E{R_n^2} = \E{(R_{n-1} + X_n + Y_n)^{2}},
\end{align*}
but $R_{n-1}$ and $X_n+Y_n$ are independent, and $\E{(X_n + Y_n)^2} = 1$. Using the recursion, $\E{R^2_n} = n$.
\end{solution}
\end{exercise}
\input{trailer.tex}
