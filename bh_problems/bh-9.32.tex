\documentclass[notitlepage,a4paper,11pt]{article}

\usepackage[no-solutions]{optional}
\usepackage{../common/preamble}
\renewcommand\sectionbreak{}
\usepackage{a4wide}
\author{Lecturers PT and PD}
\date{\today}
\setlength{\parindent}{0pt}

\setcounter{theorem}{31}
\begin{exercise} BH.9.32.
Remark: The results of this exercise are (or should be) used by nearly all software packages to control inventory levels of companies such as supermarkets and bol.com.
\begin{hint}
a. Let $Y$ be the amount purchased by the first customer that comes along, let $X$ be the rv that is 1 if the customer does indeed purchase, and 0 otherwise, and let $D$ be the size of the purchase. Why is $Y=XD$? What is $\E{X}$? What is $\E{Y\given X}$? What is $\E{Y^{2}\given X}$. You might want to use BH.9.1.


b. Let $N\sim\Pois{8\lambda}$ be the number of customers that pass by. Given $N=n$, what is $\E{S\given N}$, where $S=\sum_{i=1}^N X_iD_{i}$ is the total sales? Now use the law of total expectation. What is $\V{S\given N}$? Use Eve's law  to compute $\V S$. Bigger hint, read Example BH.9.6.1.
\end{hint}
\begin{solution}
Read the hints.

a and b. Use that $X^{2}=X$ (indicator funtion), Adam and Eve, and that $N\sim \Pois{8\lambda}$,
\begin{align*}
  \E{Y|X}&= \E{XD|X} = \E{D}\E{X|X} = \mu X, &\V{Y|X} &= \V{XD|X} = X^{2} \V{D|X} = X \sigma^2 \\
  \E Y &= \mu p, & \V Y &= \E{\V{Y|X}} + \V{\E{Y|X}} = \sigma^2 p + \mu^{2}p(1-p), \\
  \E{S|N} &= N \E Y, & \V{S|N} &= N \V Y \\
  \E N &= 8 \lambda, & \V N &=  8\lambda.
\end{align*}
Now use Example BH.9.6.1. It's just a matter of filling in.

c.
Let $Z$ be the number of customers that actually buy something.
The chicken-egg story tells us that $Z\sim \Pois{8\lambda p}$.
If we use this in Example BH.9.6.1, we can write that $S = \sum_{i=1}^{Z} D_{i}$, rather than $S=\sum_{i=1}^{N} X_{i}D_{i}$ (by the chicken-egg story, both sums have the same distribution.)
Now consider $E{S|Z}= \E D \E{Z}$, and $\V{S|Z} = \V{D}\E{Z} + (\E{D})^2 \V{Z}$.
\end{solution}
\end{exercise}
\input{trailer.tex}
