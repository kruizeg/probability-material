\documentclass[notitlepage,a4paper,11pt]{article}

\usepackage[no-solutions]{optional}
\usepackage{../common/preamble}
\renewcommand\sectionbreak{}
\usepackage{a4wide}
\author{Lecturers PT and PD}
\date{\today}
\setlength{\parindent}{0pt}

\setcounter{theorem}{37}
\begin{exercise}BH.7.38
Besides the solution of BH, read our solution.
\begin{solution}
First check~\cref{ex:3a}.

In general, I am always very careful with such `shortcuts' such as $\max\{X,Y\} + \min\{X, Y\} = X +Y$.  As a matter of fact, I try to avoid such arguments because it is easy to go wrong. Seemingly plausible arguments are often wrong due to overlooked dependency or non-linearity (effects of higher moments).

It is useful to write $\max\{x,y\} = x\1{x\geq y}+y\1{y>x}$, and something similar for the minimum. In the present case, $\cov{X,Y} = \E{X Y}-\E X \E Y$, and, similarly, $\cov{M,L} = \E{ML}- \E M \E L$, where $M$ is max, and $L$ is min. With the above indicators, it is simple to show that $\E{ML} = \E{X Y}$:
\begin{align*}
 ML
  &= (X\1{X\geq Y} + Y\1{Y\geq X})((X\1{X<Y} + Y\1{Y<X}) \\
  &= XY\1{X\geq Y} + XY\1{Y<X} = XY
\end{align*}
since $\1{X\geq Y}\1{X<Y} =0$.

However, take $X,Y\sim \Exp{\lambda}$. Then, $\E M = 3/(2\lambda)$ and $\E L= 1/(2\lambda)$, but $\E X = \E Y = 1/\lambda$.
\end{solution}
\end{exercise}
\input{trailer.tex}
