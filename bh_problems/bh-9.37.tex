\documentclass[notitlepage,a4paper,11pt]{article}

\usepackage[no-solutions]{optional}
\usepackage{../common/preamble}
\renewcommand\sectionbreak{}
\usepackage{a4wide}
\author{Lecturers PT and PD}
\date{\today}
\setlength{\parindent}{0pt}

\setcounter{theorem}{36}
\begin{exercise} BH.9.37.
 \begin{solution}
Here are some remarks.
\begin{enumerate}
\item  Bootstrapping is used in statistics to, for instance, construct confidence intervals. It is a much used and intuitive technique.
\item Extra exercise to  help you recall some ideas of Ch 1. How many different bootstrap samples are possible? (The solution is at the end.)
\item  We use some simple ideas to make the solution a bit simpler. We say that the rvs $\{X_i\}$ are independent and  distributed as the common rv~$X$ when $X_i\sim F_X$ where $F_X$ is the CDF of the rv $X$. Then $\E{X_{i}}=\E X$, and so on.  Next,  I prefer to write $Y_{j}= X_j^{*}$, as this writes (and types) faster. Finally,  it is easy to define
$Y_j = \sum_{i=1}^n X_i\1{S_j = i}$, where $S_{j}\sim \DUnif{\{1, \ldots, n\}}$ is the \(j\)th sample of the $\{X_{i}\}$.
\end{enumerate}


a. Here you should assume that the $X_i$ are not yet known. Thus, the expectation over $X_i$ is taken with respect to the CDF $F_X$. Using the independence of $X_j$ and $S_j$, $\1{S_j=i}\1{S_j=k} = 0$ if $i\neq k$, and that $\E{\1{S_j=k}} = 1/n$,
\begin{align*}
\E{Y_j} &= \sum_i \E{X_i}\E{\1{S_{j}=i}} = \mu, \\
\E{Y_j^2}  &= \E{\sum_k\sum_{l} X_k X_{l} \1{S_j=k}\1{S_{j}=l}}
= \E{\sum_k X_k^{2}  \1{S_j=k}} =\sum_{k} \E{X^{2}} n^{-1} = \E{X^{2}}, \\
\V{Y_j}&= \E{Y_j^{2}}- (\E{Y_{j}})^{2} = \sigma^{2}.
\end{align*}

b. Now we are given the outcomes (samples) $X_i=x_i$ of $n$ experiments.
I prefer to write $D = X_{1}, \ldots, X_{n}$ as it is shorter.
Noting that $S_j$ and $D$ are independent, and that $\E{X_{k}|D} = X_{k}$,
\begin{align*}
  \E{Y_{j}|D} = \sum_k X_k \E{\1{S_j=k}|D} = \frac 1 n \sum_k X_k := \bar X.
\end{align*}
By symmetry, this holds for all $j$.
Observe that this sample average $\bar X$  is in general not the same as $\mu$!
Next,
\begin{align*}
\E{\bar Y|D} &= \E{\frac 1 n\sum_{j}Y_{j}|D} = \frac 1{n} \sum_j \E{Y_j|D} = \frac 1 n n \E{Y_1|D} = \bar X.
\end{align*}


Now the conditional variance. Since $S_j$ and $S_k$ are independent when $j\neq k$, it must be that $Y_j|D$ and $Y_k|D$ are also conditionally independent for all $j\neq k$. Therefore,
\begin{align*}
\E{Y_{j}^{2}|D}
&= \E{\sum_k\sum_l X_kX_l\1{S_j=k}\1{S_j=l}|D} \\
&= \E{\sum_kX_k^{2}\1{S_j=k}|D}  = \sum_kX_k^{2}\E{\1{S_j=k}|D} \\
&= \frac 1 n \sum_{k} X_k^{2}, \\
\V{Y_j| D} &= \E{Y_j^2|D} - (\E{Y_j|D})^{2} = \frac 1 n \sum_{k} X_k^{2} - (\bar X)^{2} = \frac 1 n \sum_{k} (X_k - \bar X)^{2}.
\end{align*}
As this holds for all $j$, and the $Y_{j}|D$ are conditionally independent,
\begin{align*}
\V{\bar Y|D}
=\V{\frac 1 n\sum_{j}Y_{j}|D} = \frac 1{n^{2}} \sum_j \V{Y_j|D} = \frac 1 n \V{Y_1|D}.
\end{align*}


c. For $\E{\bar Y}$ use Adam's law, linearity and symmetry:
\begin{align*}
\E{\bar Y} = \E{\E{\bar Y|D}} = \E{\bar X} = \E{\frac 1 n \sum_{k=1}^{n}X_{k}} =  \frac 1 n \sum_{k}\E{X_{k}} = \E X = \mu.
\end{align*}


Here are the details for $\V{\bar Y}$. Using  examples BH.6.3.3 and BH.6.3.4,
\begin{align*}
\E{\V{\bar Y|D}} &=  \frac 1 n \E{\V{Y_{1}|D}} =
\frac 1 {n^2} \E{ \sum_{i=1}^n (X_i-\bar X)^2} \\
&= \frac{n-1} {n^2} \E{\frac 1 {n-1} \sum_{i=1}^n (X_i-\bar X)^2}
= \frac{n-1} {n^2} \E{S_{n}^2} = \frac{(n-1)\sigma^{2}} {n^2} \\
\V{\E{\bar Y|D}} &= \V{\bar X}= \frac 1{n^{2}} \sum_{i} \V{X_{i}} = \frac 1 n \sigma^{2}.
\end{align*}
Now use Eve's law to add both terms to get $V{\bar Y}$:
\begin{align}
  \V{\bar Y} = \E{\V{\bar Y|D}} + \V{\E{\bar Y|D}} = \frac{n-1}{n^{2}}\sigma^2 + \frac{1}{n} \sigma^2 = \frac{2n-1}{n^{2}}\sigma^{2}.
\end{align}


d. We add randomness twice, first we draw  samples to get $D$, and then we draw randomly from $D$.

The extra exercise: immediate from Example 1.4.22. We are not interested in the sequence of the bootstrap sample. BTW, the story that goes for me with this example is the `balls and bars story'. I have $n$ balls to distribute over $k$ boxes. Hence, there are $k-1$ bars to separate the boxes. For the bootstrap sample, I have to distribute $n$ bootstrap samples (the $X^*_{i}$) over $n$ boxes (the initial sample $X_i$.)

If $n$ is small, say $n=4$. Does it make sense to take more than 1000 bootstrap samples?
\end{solution}
\end{exercise}
\input{trailer.tex}
