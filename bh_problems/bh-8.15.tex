\documentclass[notitlepage,a4paper,11pt]{article}

\usepackage[no-solutions]{optional}
\usepackage{../common/preamble}
\renewcommand\sectionbreak{}
\usepackage{a4wide}
\author{Lecturers PT and PD}
\date{\today}
\setlength{\parindent}{0pt}

\setcounter{theorem}{14}
\begin{exercise} BH.8.15
We'll use this exercise in a lecture to show how the normal distribution originates from astronomy (or dart throwing).

The notation is a bit clumsy for the angle coordinate. Write $\Theta$ for the rv and $\theta$ for its value.
\begin{hint}
a. See BH.8.1.9.

b. If $(X,Y)$ are uniform on the disk, then the function $g(x,y)$ must be constant on this disk. Use an indicator to ensure that $X^2+Y^2\leq 1$. Finally, normalize.

c. What are the densities of $X$ and $Y$ when they are $N(0,1)?$
\end{hint}

\begin{solution}
a.  I remember this: $f_{X,Y}(x,y) \d x \d y = f_{R, \Theta}(r, \theta) \d r \d \theta$. From this,
\begin{align*}
f_{R, \Theta}(r, \theta)  = f_{X,Y}(x,y) \left| \frac{\partial (x,y)}{\partial(r, \theta)} \right|.
\end{align*}
Now, since $x=r\cos \theta$ and $y=r \sin \theta$,
\begin{align*}
\frac{\partial (x,y)}{\partial(r, \theta)} =
  \begin{pmatrix}
    \cos \theta & -r \sin \theta \\
 \sin \theta & r \cos \theta
  \end{pmatrix},
\end{align*}
which has determinant equal to $r$.
It is given that $f_{X,Y}(x,y)=g(x^2+y^2) = g(r^2)$. Hence,
\begin{align*}
f_{R, \Theta}(r, \theta)  = f_{X,Y}(x,y) r = g(r^2)r,
\end{align*}
with $r\geq 0, \theta\in[0, 2\pi$.
The RHS  does not  depend on $\theta$. Hence, $f_{\Theta}(\theta)$ must be a constant.

b. Use the hint. Since $g$ is a constant, $f_{R, \Theta}(r, \theta) \propto r$. Thus,
\begin{align*}
  \int_{0}^{1}\int_{0}^{2\pi} r \d r \d \theta = 2\pi (1/2) r^{2}|_0^{1} = \pi.
\end{align*}
So, $1/\pi$ is the normalization constant.

c. $f_{X,Y}(x,y) = \exp{-x^2}/\sqrt{2\pi}\exp{-y^{2}}/\sqrt{2\pi} = \exp{-(x^2+y^2)/2\pi} = \exp{-r^{2}}/2\pi$. Indeed, $f_{X,Y}(x,y)$ has the form $g(x^2+y^2)$. The rest is as in part b.
\end{solution}
\end{exercise}
\input{trailer.tex}
