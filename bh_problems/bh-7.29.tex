\documentclass[notitlepage,a4paper,11pt]{article}

\usepackage[no-solutions]{optional}
\usepackage{../common/preamble}
\renewcommand\sectionbreak{}
\usepackage{a4wide}
\author{Lecturers PT and PD}
\date{\today}
\setlength{\parindent}{0pt}

\setcounter{theorem}{28}
\begin{exercise} BH.7.29
\begin{solution}
a. For the independence of $L$ and $M$, when $i>j$ then $\P{L=i, M=j}=0$, but $\P{L=10}\P{M=9} > 0$. Check the definition of independence of rvs to see why this forms a counter example.

This must be clear: for $i\leq j$,
\begin{align*}
\P{L=i, M=j}
  &= \P{X=i, Y=j} + \P{X=j, Y=j} - \P{X=Y=i}\\
  &\stackrel{2}{=} 2\P{X=i, Y=j} - \P{X=Y=i} \\
  &\stackrel{3}{=} 2 p^{2} q^{i+j} - p^{2}q^{2i}\\
\end{align*}
where step $2$ by symmetry, and step 3 by indepence (recall, this was given in the problem description) and the PMF of a $\Geo{p}$ rv.

b. We use marginalization. Look up how to deal with $\sum_{i=0}^{\infty} q^{i}$ in the appendix.
\begin{align*}
  \P{L=i}
  &= \sum_{j\geq i} \P{L=i, M=j} \\
  &= 2p^2q^i\sum_{j> i} q^j + p^2q^{2i} \\
  &= 2p^2q^i q^{j+1}\sum_{k\geq 0} q^k + p^2q^{2i} \\
  &= 2p^2q^i q^{i+1}\frac{1}{1-q} + p^2q^{2i} \\
  &= 2p q^{2i+1} + p^2q^{2i} \\
  &= (2 q + p)pq^{2i} \\
  &= (q + 1)pq^{2i} \\
  &= (q + 1)(1-q)q^{2i} \\
  &= (1-q^{2})q^{2i}.
\end{align*}
So, from this form we see that $L\sim\Geo{1-q^{2}}$. Why is this? Well, the event $\{L\geq i\} = \{X\geq i\}\{Y\geq i\}$. But $\P{X\geq i} = q^{i}$ (Do the algebra, which is similar to the bunch of summations just above), and so $\P{L\geq i} = q^{2i}$, which implies $L\sim \Geo{1-q^{2}}$.

c. With the hint this is simple, because $\E{M} = \E{X}+\E{Y}- \E{L}$. Now $\E X = \E{Y} = q/p$, and since $L\sim \Geo{1-q^2}$, we have that $\E L = q^2/(1-q^{2})$. Thus,
\begin{equation*}
\E{M} = 2 q/ p - q^2/(1-q^2).
\end{equation*}

d. Using a.)
\begin{align*}
  \P{L=i, M-L=k} = \P{L=i, M=k+i} = 2p^2q^{2i} q^k - p^2q^{2i}=p^2q^{2i}(2q^k-1).
\end{align*}
So, if $k=0$ we have
Clearly this splits into a product of a term with just an $i$ and another term with just a $k$, thereby implying that $L$ and $M-K$ are indepedent.
\end{solution}



\end{exercise}
\input{trailer.tex}
