\documentclass[notitlepage,a4paper,11pt]{article}

\usepackage[no-solutions]{optional}
\usepackage{../common/preamble}
\renewcommand\sectionbreak{}
\usepackage{a4wide}
\author{Lecturers PT and PD}
\date{\today}
\setlength{\parindent}{0pt}

\setcounter{theorem}{23}
\begin{exercise}BH.7.24
In the assignments we'll develop a simulator.

\begin{hint}
Check BH.7.1.24 and BH.7.1.25
First draw the area over which we have to integrate. Then use an indicator function over which to integrate. What is the joint PDF  $f_{Y_1, Y-2}$?
\end{hint}
\begin{solution}
a. From the hint,
\begin{align*}
\P{Y_1<c Y_2}
  &= \int \int \1{x<c y}\lambda_1 e^{-\lambda_1 x} \lambda_2e^{-\lambda_2 y}\d x \d y
  = \lambda_1\lambda_2\int_0^{\infty} e^{-\lambda_{1}x}\int_{x/c}^{\infty} e^{-\lambda_{2} y}\d y \d x\\
  & = \lambda_1\int_0^{\infty} e^{-\lambda_{1}x} e^{-\lambda_{2} x/c} \d x
  = \frac{\lambda_1}{\lambda_1+\lambda_2/c}.
\end{align*}
Check the result for $c=0$ and $c=\infty$.

I prefer to use conditioning, like this:
\begin{align*}
\P{Y_1<c Y_2}
  &= \int \P{Y_1<cY_2| Y_1=x}\lambda_1 e^{-\lambda_1 x} \d x
  = \int \P{Y_2>x/c| Y_1=x}\lambda_1 e^{-\lambda_1 x} \d x\\
&= \int e^{-\lambda x/c} \lambda_1 e^{-\lambda_1 x} \d x,
\end{align*}
and the rest goes as before. Actually, I tend to use conditioning as it helps to make the reasoning easier. In this case, suppose that I know that $Y_1=x$, what can I say about $\P{Y_2 > c x}$?

BTW, conditioning does not always make things simpler. When rvs are dependent, then you have to watch out.

b. See the solutions of BH on the web.
\end{solution}
\end{exercise}
\input{trailer.tex}
