\documentclass[notitlepage,a4paper,11pt]{article}

\usepackage[no-solutions]{optional}
\usepackage{../common/preamble}
\renewcommand\sectionbreak{}
\usepackage{a4wide}
\author{Lecturers PT and PD}
\date{\today}
\setlength{\parindent}{0pt}

\setcounter{theorem}{57}
\begin{exercise} BH.7.58.
Remark: this is a totally great exercise. First solve it yourself. In the solution, I'll explain why, in particular how to relate the concept of covariance to the determinant of a matrix.

\begin{hint}
a. Expand the brackets in the expression for the sample variance $r$ to see that
\begin{align*}
r = 1/n \sum_i x_i y_i - \bar x \bar y.
\end{align*}
Next, we choose with probability $1/n$ one the points $(x_i, y_{i})$.  Under this probability, $\E{X Y} = 1/n \sum_i x_i y_i$, $\E X = \bar x, \E Y = \bar y$. So, how do $\cov{X,Y}$ and $r$ relate?


b. Expand the brackets and use iid and linearity properties to show that the expected area spanned by two random points $(X,Y)$ and $(\tilde X, \tilde Y)$ satisfies
\begin{align*}
\E{(X-\tilde X)(Y-\tilde Y)} = 2\cov{X,Y}.
\end{align*}

\end{hint}
\begin{solution}

b. Use the hint. Then, if we choose two points at random from the sample, then $(x_i-x_j)(y_i-y_j)$ is the area spanned by these  two points.
More generally, I have $n$ choices for my first point, and also $n$ choices for the second point (if both points are the same, the area of the rectangle is 0, so we don't have to exclude such choices).
Hence, the expected area of the rectangle spanned by the two random points $(X,Y)$ and $(\tilde X, \tilde Y)$ is
\begin{align*}
\frac 1 {n^2} \sum_{i,j} (x_i-x_j)(y_i-y_j).
\end{align*}
Simplify this to show that
\begin{align*}
2 \frac 1 n \sum_i x_i y_i - 2 \bar x \bar y = 2 r
\end{align*}
Hence, by part a., the expected area is twice the covariance.

Why is $\cov{X,a}= 0$ for $a$ a constant? Because the `area' of rectangles, all with the same \(y\)-coordinate, is zero, i.e., they lie on a line.

c.  This is the part of the exercise that explains what the above is all about.
Since there is a direct relation between covariance and area, we can use geometric arguments to derive (and memorize!) all properties of covariance! Write property i. of covariance  as $\cov{X,Y} = \cov{Y,X}$. Suppose I flip the \(x\) and \(y\)-axis, does the area of a rectangle change?  For property ii., what happens to the area of rectangle if you stretch the sides? For property iii., realize that this is just a shift of a rectangle that leaves its area invariant. For property iv., what happens to the area if you put an extra rectangle on top or to the right?

BTW, property iii. follows directly from property iv. In iv., take $W_3$ equal to a constant $a_2$, in other words $\P{W_3=a_2}=1$. We know that $\cov{X, a} = 0$ for a constant $a$.

Here are  some final remarks.

Let's put all the above in a very general frame.  The covariance has a number of interesting properties:
\begin{enumerate}
\item  It is bilinear, that is, the covariance is linear in both arguments. The linearity in the first argument means that $\cov{X+Y, Z}=\cov{X,Z}+\cov{Y,Z}$ and $\cov{a X, Z}=a\cov{X,Z}$ for $a\in \R$. The linearity in the second argument means that $\cov{X, Y+Z}=\cov{X,Y}+\cov{X,Z}$ and $\cov{X, a Z}=a\cov{X,Z}$ for $a\in \R$.
\item It is symmetric: $\cov{X, Y}=\cov{Y,X}$, from which we define $\V X = \cov{X,X}$.
\item  $\cov{X,a} = 0$ for all $a\in R$.
\end{enumerate}
If you memorize the first two properties of covariance, all the rest follows.

Now we do some geometry. Take three vectors $x,y, z\in \R^2$ (it's easy to generalize to $\R^n)$. Then we know that the area $D(x,y)$ of the parallelogram spanned by vectors $x$ and $y$  satisfies the following properties.
\begin{enumerate}
\item  Area is bilinear. The linearity in the first argument means that $D(x+y, z) = D(x, z) + D(y, z)$ and $D(ax, z)=a D(x, z)$ for $a\in \R$. (Just make a drawing to convince you about this.) The linearity in the second argument means that  $D(x, y+z) = D(x, y) + D(x, z)$  and $D(x, a z)=a D(x, z)$ for $a\in \R$.
\item  $D(x,x)$ = 0; there is no area between $x$ and $x$.
\item $D( (1,0), (0,1)) = 1$; the area of the square with side 1 is 1.
\end{enumerate}
In fact, thex first property means that stretching vectors and stacking parallelograms result in stretching and adding areas.
The second says that the area of a parallelogram spanned by two parallel vectors is zero. The third specifies that the area of the unit square is 1.

Now it can be proven that there exists just one function $D$ that satisfies these properties. In fact, this is the determinant of the matrix with as columns the vectors that span the parallellogram. Moreover, it can be shown that the second property can be replaced by the skew-symmetric property: $D(x,y) = - D(x,y)$.
(Note that $D(x,x) = -D(x,x) \implies 2 D(x,x) = 0 \implies D(x,x) = 0$.)

Let us use the properties to compute the area of a parallelogram spanned by the vectors $x = (a,b)$ and $y =(c,d)$ in 2D. Then
\begin{align*}
D(x,y) &= D((a,b), (c,d)) =  D(a (1,0)) + b(0,1), c(1,0) + d(0,1)) \\
&= ad D((1,0), (0,1)) + b c D((0,1), (1,0)) = ad - b c,
\end{align*}
where we use bilinearity in the first step, and skew-symmetry in the second and third. And this is indeed the determinant of the matrix with $x$ and $y$ as columns.

So, all in all, this is what I remembered throughout the years: the covariance and the determinant are bi-linear forms, the first is symmetric, the second skew- (or anti-)symmetric.

Finally, I don't see why the areas of the rectangles have to have a sign in this problem. Interestingly, for the determinant, the areas of the parallelograms do have to have a sign to make the concept useful for physics.
\end{solution}
\end{exercise}
\input{trailer.tex}
