\documentclass[notitlepage,a4paper,11pt]{article}

\usepackage[no-solutions]{optional}
\usepackage{../common/preamble}
\renewcommand\sectionbreak{}
\usepackage{a4wide}
\author{Lecturers PT and PD}
\date{\today}
\setlength{\parindent}{0pt}

\setcounter{theorem}{58}
\begin{exercise}BH.7.59
Read this exercise, then read (and do) BH.5.53 for some further background.
You'll encounter these topics countless times in other courses!
The final answer is really nice and intuitive.

\begin{hint}
a. Use that expectation is linear.

b. Read the entire exercise in its entirety before trying to solve it. In this case trying to solve c.\/ seems simpler because of the extra iid assumption. You  might want to use this to formulate some simple guesses.

Thus first part c. It is given that the $X_i$ and $Y_j$ are iid. Then, if I could improve the estimator $\hat \theta$ by splitting the measurements into two sets $X_i$ and $Y_j$, then I would certainly do that.
And not only I would do that; anybody in his right mind would do that.
But, I never heard of this idea, and I am sure you have neither, so this must be impossible (because if it would, people would have been using this trick for ages.)
Hence, we can place this in the context of the maxim: `we cannot obtain information for free'.
For this case, this must imply that splitting iid measurements into smaller sets cannot help with improving the estimator. What does this idea imply for the weights?


Part b, continued. I always try to solve the problem myself without a hint. This lead to the following considerations, which gave me quite a bit of extra understanding beyond the problem itself.  As a next piece of advice, before doing hard work, I prefer to look at some corner cases to acquire some intuitive understanding. I also use the rvs of Part c.

Suppose  that $v_2:=\V{Y_j} = 0$, but $v_1 := \V{X_i} > 0$. (For instance, $Y_j$ is the $j$th measurement of a perfect machine and $X_j$ of an imperfect machine.)
Then we know that the set $\{Y_j\}$ forms a set of perfect measurements.
But then I am not interested in the $\{X_i\}$ measurements anymore; why should I as I have the perfect measurements $\{Y_j\}$ at my disposal.
So, then I put $w_1=0$, because I don't want the $\{X_i\}$ measurements to pollute my estimator.
In other words, the final result should be such that $v_2=0 \implies w_{1}=0$, and vice versa.


More generally, I learned from this  corner case that I want this for the final result:  when $v_2<v_{1} \implies w_1 < w_{2}$, and vice versa.

How would you choose the weights such that this requirement is satisfied, but also the condition imposed by Part c.?
\end{hint}


\begin{solution}
a. Follows directly from the hint.

Check the hint!

c.  If $X_i$ and $Y_j$ are iid, it must be that $w_{1} = n/(n+m)$.

b. Can we make some further progress, just by keeping a clear mind?
Well, in fact we can by using our insights of part c.
If we have $n+m$ iid measurements of which we call $n$ measurements of type $X_i$, and $m$ of type $Y_{j}$, then
\begin{align*}
\V{\hat \theta_{1}}  =  \E{\left(\frac{1}{n}\sum_{i}X_{i} - \theta\right)^2} = n^{-2}\E{\left(\sum_i (X_{i}-\theta)\right)^{2}} = n^{-2}\V{\sum_i X_i} = \V{X_1}/n = \sigma^{2}/n.
\end{align*}
So, $n=\sigma^{2}/\V{\hat \theta_1}$, and likewise $m=\sigma^{2}/\V{\hat \theta_2}$. Finally, plug this into our earlier expression for $w_1$ to  get
\begin{align*}
w_1 = \frac n {n+m} = \frac{\sigma^2/\V{\hat \theta_1}}{\sigma^2/\V{\hat \theta_1} + \sigma^2/\V{\theta_2}} = \frac{\V{\hat \theta_2}}{\V{\hat \theta_1} + \V{\theta_2}}.
\end{align*}
If we check our earlier insight, then we see that if $\V{Y_j}=0$, then $\V{\theta_2}=0$, hence $w_1=0$ in that case. This is precisely what we wanted.

Let us finally use the  hint of BH to check that the above expression for $w_1$ is correct.
\begin{align*}
\E{(\hat \theta -\theta)^{2}} =
\E{(w_1(\hat \theta_{1} - \theta) + w_2(\hat \theta_{2}-\theta))^{2}} =
\V{w_1 \hat \theta_{1}} + \V{w_2 \hat \theta_{2}},
\end{align*}
by independence. Take the $w$'s out of the variances, then write $w_2=1-w_1$, take $\partial_{w_1}$ of the expression,  set the result to 0, and solve for $w_1$. You'll get the above expression.
\end{solution}
\end{exercise}
\input{trailer.tex}
