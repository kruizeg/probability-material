\documentclass[notitlepage,a4paper,11pt]{article}

\usepackage[no-solutions]{optional}
\usepackage{../common/preamble}
\renewcommand\sectionbreak{}
\usepackage{a4wide}
\author{Lecturers PT and PD}
\date{\today}
\setlength{\parindent}{0pt}

\setcounter{theorem}{25}
\begin{exercise} BH.10.26
\begin{solution}
a.
I did things a bit differently than in the book. Take $S_n=\sum_{i=1}^n X_i$ with $X_i\sim\Bern{p}$. Then I know this:
\begin{align*}
\P{S_n=k} = {n \choose k} p^k(1-p)^{n-k} \to e^{-\lambda} \lambda^k/k! = \P{N=k}, \quad\text{if } N\sim\Pois{\lambda},
\end{align*}
for $n\to \infty$, $p\to0$ but such that $p n = \lambda$.
I also know from the CTL that $S_n\sim N(n p, n p(1-p))$ if $n$ becomes large.
But, $N(n p, n p(1-p)) \to N(\lambda, \lambda)$ in the above limit.
Now take $\lambda=n$ to see that $\Pois{\lambda} \sim N(n,n)$.

b.
Check the solution manual. Then, with $\mu=\sigma=\lambda=n$, and $n\gg 1$,
\begin{align*}
\Phi(n+1/2) - \Phi(n-1/2)
&= \frac 1 {\sqrt{2\pi \sigma^{2}}}\int_{n-1/2}^{n+1/2} e^{-(x-\mu)/2\sigma^{2}} \d x\\
&= \frac 1 {\sqrt{2\pi n }}\int_{-1/2}^{1/2} e^{-x^{2}/2n} \d x\\
&= \frac 1 {\sqrt{2\pi n }}\int_{-1/2}^{1/2} (1 - x^{2}/2n) \d x\\
&= \frac 1 {\sqrt{2\pi n }} (1- 1/(24 n)).
\end{align*}
So, we found another term to approximate $n!$ yet better.
\end{solution}
\end{exercise}
\input{trailer.tex}
