% arara: pdflatex: { shell: yes }
% arara: pythontex: {verbose: yes, rerun: modified }
% arara: pdflatex: { shell: yes }
% arara: move: {  files: ['study-guide.pdf'], target: ['../pdf-files/'] }

\documentclass[a4paper,11pt]{article}


\usepackage[all-solutions-at-end]{optional}

\usepackage{../common/preamble}
%\newcommand{\sectionbreak}{} %\clearpage}


\newtheorem{theorem}{Theorem}
\newtheorem{exercise}[theorem]{Ex}
\newtheorem{remark}[theorem]{Remark}

\setcounter{tocdepth}{1}
\usepackage{a4wide}
\usepackage{../common/abbreviations}

\author{Lecturers PT and PD
}
\date{\today}
\title{Probability Distributions, Week 6}

\opt{all-solutions-at-end}{
\Opensolutionfile{hint}
\Opensolutionfile{ans}
}


\begin{document}
\maketitle
\tableofcontents


\section{Lecture 11}
\label{sec:lecture-1}


\begin{itemize}
\item 10.1: read, but skip examples 10.1.7, 10.1.8, 10.1.9
\item BH video: 28
\item BH Exercises: 10.2, 10.3, 10.6, 10.9,
\end{itemize}



\begin{exercise}
We have a continuous r.v. $X$ with PDF
\begin{align}
  \label{eq:1123}
  f(x) =
  \begin{cases}
    1/2, & 0\leq x < 1, \\
    1/4, & 1\leq x \leq 3,
  \end{cases}
\end{align}
and $0$ elsewhere. Compute $\E X$ and $\V X$.

Then define the r.v.
$Y$ such that $Y=\1{X< 1} + 2\1{X\geq 1}$.
Explain what information you obtain about $X$ when somebody tells that $Y=1$, or that $Y=2$.
Use conditioning on $Y$ and Adam's and Eve's laws to recompute $\E X$ and $\V X$.

Again, with this exercise we can check whether we are applying all concepts in the correct way.
\begin{solution}

Using the properties of the uniform distribution,
\begin{align}
\E X &= \frac{1}{2} \int_0^1 x \d x + \frac{1}{4} \int_1^3 x \d x = \frac{1}{4} + \frac{9-1}{8} = \frac{5}{4} \\
\E{X^2} &=  \frac{1}{2}\int_0^{1} x^{2}\d x + \frac{1}{4}\int_1^{3} x^{2}\d x = \frac{1}{6} + \frac{27-1}{12} = \frac{7}{3} \\
\V X &=  \frac{7}{3} - \frac{25}{16} = \frac{37}{48}.
\end{align}

$Y=1 \implies 0\leq X< 1$,  $Y=2 \implies 2\leq X \leq 4$.

\begin{align}
\E{X\given Y=1} &= 1/2, \\
\E{X\given Y=2} &= 2, \\
g(y) &= \E{X\given Y=1}\1{y=1} + \E{X\given Y=2}\1{y=2}\\
g(Y) &= \E{X \given Y} = \frac{1}{2} \1{Y=1} + 2 \1{Y=2},\\
\E{X} &= \E{\E{X\given Y}} = \E{g(Y)} \\
&=\E{\frac{1}{2} \1{Y=1} + 2 \1{Y=2}} \\
&= \frac{1}{2} \E{\1{Y=1}} + 2 \E{\1{Y=2}} \\
&= \frac{1}{2} \P{Y=1} + 2 \P{Y=2} \\
\P{Y=1} &= \E{\1{Y=1}} = \E{\1{X<1}} = \frac{1}{2} \\
\P{Y=2} &= \E{\1{Y=2}} = \E{\1{X\geq 1}} = \frac{1}{2} \\
\E X &= \frac{1}{2} \frac{1}{2} + 2 \frac{1}{2} = \frac{5}{4}.
\end{align}


\begin{align}
\V{X\given Y} &= \V{X\given Y=1}\1{Y=1} + \V{X\given Y=2}\1{Y=2} \\
&= \frac{1}{12} \1{Y=1} + \frac{4}{12} \1{Y=2}, \\
\E{\V{X\given Y}} &= \frac{1}{12} \frac{1}{2} + \frac{4}{12} \frac{1}{2} = \frac{5}{24}, \\
\V{\E{X \given Y}} &= \frac{1}{4}\V{\1{Y=1}} + 4\V{\1{Y=2}} + 2\cov{\frac{1}{2}\1{Y=1}, 2\1{Y=2}} \\
\V{\1{Y=1}} &= \V{\1{Y=2}}  = \frac{1}{4}, \\
2\cov{\frac{1}{2}\1{Y=1}, 2\1{Y=2}} &= 2\cov{\1{Y=1}, \1{Y=2}}=-2\E{\1{Y=1}}\E{\1{Y=2}} = -\frac{1}{2}\\
\V{\E{X \given Y}} &= \frac{1}{16} + 1 - \frac{1}{2} = \frac{9}{16}\\
\V X &= \E{\V{X\given Y}} + \V{\E{X\given Y}} = \frac{5}{24} + \frac{9}{16}.
\end{align}
I forgot the covariance at first :-(
\end{solution}
\end{exercise}


\section{Lecture 12}

\begin{itemize}
\item 10.2: read
\item BH video: 29
\item BH Exercises:
\end{itemize}


\begin{exercise}
A server (e.g., a machine, a mechanic, a doctor) spends a random amount $T$ on a job.
While the server works on the job, it sometimes gets interrupted to do other tasks $\{R_{i}\}$.
Such tasks can be failures of the machine, and they need to be repaired before the machine can continue working again.
(In the case of a mechanic, interruptions occur when the mechanic has to check some other machine or help other mechanics.)
It is clear that such interruptions can only occur when the server is working.
Assume that the interruptions arrive according to a Poisson process with rate $\lambda$, i.e., the number of failures $N| T \sim\Pois{\lambda T}$.
The interruptions $\{R_{i}\}$ are independent of $T$ and form an iid sequence with common mean $\E R$ and variance $\V R$.
(Typically, we get estimates for $\E T$ and $\V T$ from measurements.)

Use Adam and Eve's laws to express the expectation and variance of the total time $S$ to complete a job in terms of $\E T$ and $\V R$.
\begin{solution}

Below we need some properties of the Poisson distribution.
\begin{align}
\E{N(T)} &= \E{\E{N(T)\given T}}\\
\E{\E{N(T)\given T=t}}&= \E{N(t)} = \lambda t \\
\E{\E{N(T)\given T}}&= \lambda T \\
\E{N(T)} &= \E{\E{N(T)\given T}} = \E{\lambda T} = \lambda \E T\\
\V{N(t)} &= \lambda t
% \V N &= \E{\V{N\given T}} + \V{\E{N\given T}} \\
% \V{N\given T} &= \lambda T  \implies \E{\V{N\given T}} = \lambda \E{T} \\
% \E{N\given T} &= \lambda T \implies \V{\E{N\given T}} = \V{\lambda T} = \lambda^{2} \V T \\
% \V N &= \lambda \E T + \lambda^{2} \V T.
\end{align}

The  duration $S$ of a job  is its own service time $T$ plus all interruptions, hence $S = T+ \sum_{i=1}^{N(T)} R_{i}$, hence
\begin{align}
\E S &= \E{\E{S\given T}}   = \E{\E{T + \sum_{i=1}^{N(T)} R_{i}\given T}} \\
\E{S\given T = t} &= \E{T + \sum_{i=1}^{N(T)} R_{i}\given T=t} = \E{t + \sum_{i=1}^{N(t)} R_{i}} = t + \E{\sum_{i=1}^{N(t)} R_{i}} \\
\E{\sum_{i=1}^{N(t)} R_{i}}  &= \E{\E{\sum_{i=1}^{N(t)} R_{i}\given N(t)}}\\
\E{\sum_{i=1}^{N(t)} R_{i}\given N(t)= n} &= \E{\sum_{i=1}^{n} R_{i}} = n \E R  \implies \E{\sum_{i=1}^{N(t)} R_{i}\given N(t)} = N(t) \E R \\
\E{\sum_{i=1}^{N(t)} R_{i}} &= \E{N(t) \E R} = \E R \E{N(t)} = \lambda t \E R\\
\E{S \given T=t} &= t + \lambda \E R t = (1+\lambda \E R) t \\
\E{S \given T} &= (1+\lambda \E R) T \\
\E S  &= \E{(1 + \lambda \E R )T} = (1+\lambda \E R) \E T.
\end{align}
(When rereading this exercise, explain per step which assumption we use to get to the next step.)

For the variance,
\begin{align}
\V S &= \E{\V{S\given T}} + \V{\E{S\given T}}.
\end{align}
From the above we immediately have
\begin{align}
\V{\E{S\given T}} &= \V{(1+\lambda \E R) T} = (1+\lambda\E R)^{2} \V T.
\end{align}
For $\E{\V{S\given T}}$ we go step  by step.
\begin{align}
\V{S\given T=t} &= \V{T + \sum_{i=1}^{N(T)} R_{i}\given T=t} = \V{\sum_{i=1}^{N(t)} R_{i}},
\end{align}
as $\V t = 0$. But $N(t)$ is a Poisson rv. So we need to use Eve's law again, but now condition on $N(t)$.
\begin{align}
\V{\sum_{i=1}^{N(t)} R_{i}} &= \E{\V{\sum_{i=1}^{N(t)} R_{i}\given N(t)}} + \V{\E{\sum_{i=1}^{N(t)} R_{i}\given N(t)}}.
\end{align}
For the middle part:
\begin{align}
\V{\sum_{i=1}^{N(t)} R_{i}\given N(t)=n} &= \V{\sum_{i=1}^{n} R_{i}}  = n \V R \\
\V{\sum_{i=1}^{N(t)} R_{i}\given N(t)} &= N(t) \V R \\
\E{\V{\sum_{i=1}^{N(t)} R_{i}\given N(t)}} &= \E{N(t) \V R}=  \lambda t \V R.
\end{align}
For the right part:
\begin{align}
\E{\sum_{i=1}^{N(t)} R_{i}\given N(t)=n} &= n \E R\\
\E{\sum_{i=1}^{N(t)} R_{i}\given N(t)} &= N(t) \E R\\
\V{\E{\sum_{i=1}^{N(t)} R_{i}\given N(t)}} &= \V{N(t)\E R} = (\E R)^{2} \V{N(t)} = (\E R)^{2} \lambda t.
\end{align}
And so,
\begin{align}
\V{\sum_{i=1}^{N(t)} R_{i}} &= \lambda t \V R  + \lambda t (\E R)^{2}  = \lambda t \E{R^{2}}.
\end{align}
Putting the things together:
\begin{align}
\V{S\given T=t} &= \V{\sum_{i=1}^{N(t)} R_{i}} = \lambda t \E{R^{2}}\\
\V{S\given T} &= \lambda T \E{R^{2}}\\
\E{\V{S\given T}} &= \lambda \E{R^{2}} \E T.
\end{align}
So, finally,
\begin{align}
\V S &= \lambda \E{R^{2}} \E T + (1+\lambda\E R)^{2} \V T.
\end{align}

Special cases for you to sort out:
\begin{enumerate}
\item  $R$ and $T$ constant. Why is still $\V S > 0$?
\item Take $R$ and $T$ exponential; then the expressions for $\V T$ etc are still easy to manage.
\end{enumerate}
\end{solution}
\end{exercise}


\section{Tutorial}
\label{sec:tutorial}


\subsection{Practice}



\begin{exercise}
$\E{XY}^2 \quad?\quad \E{X^2}\E{Y^2}$.
\begin{solution}
By the Cauchy-Schwartz inequality,
\begin{align}
    \E{XY} \leq \sqrt{\E{X^2}\E{Y^2}}.
\end{align}
Squaring both sides yields the solution
\begin{align}
    \E{XY}^2 \leq \E{X^2}\E{Y^2}.
\end{align}
\end{solution}
\end{exercise}


\begin{exercise}
$\E{\log (X) } \quad?\quad \log(\E{X})$
\begin{solution}
By Jensen's inequality, $\E{\log X } \leq \log \E{X}$, since $\log(\cdot)$ is a concave function.
\end{solution}
\end{exercise}

\begin{exercise}
$\V{Y} \quad?\quad \E{\V{Y|X}}$
\begin{solution}
By Eve's law we have
\begin{align}
    \V{Y} &= \E{\V{Y|X}} + \V{\E{Y|X}}.
\end{align}
Since $\V{\E{Y|X}} \geq 0$, it follows that
\begin{align}
    \V{Y} &\geq \E{\V{Y|X}}.
\end{align}
\end{solution}
\end{exercise}

\begin{exercise}
$\E{|X|} \quad?\quad \sqrt{\E{X^2}}$
\begin{solution}
Note that $|X| = \sqrt{X^2}$. Define $Y = X^2$ and note that $g(y) = \sqrt{y}$ is a concave function. Hence, by Jensen's inequality,
\begin{align}
    \E{|X|} &= \E{g(Y)} \leq g(\E{Y}) = \sqrt{\E{X^2}}.
\end{align}
Hence, the solution is $\E{|X|} \leq \sqrt{\E{X^2}}$
\end{solution}
\end{exercise}

\begin{exercise}
$\P{X^2 \geq 4} \quad?\quad \E{|X|}/2$
\begin{solution}
We have
\begin{align}
    \P{X^2 \geq 4} = \P{|X| \geq 2} \leq \E{|X|}/2,
\end{align}
by Markov's inequality.
\end{solution}
\end{exercise}



\begin{exercise}
Let $Z \sim N(0,1)$. Then, $\P{Z > \sqrt{2}} \quad?\quad 1/e$.
\begin{solution}
By Chernoff's inequality,
\begin{align}
    \P{Z > \sqrt{2}} &\leq \frac{\E{e^{tZ}}}{e^{\sqrt{2}t}} \\
    &= \frac{e^{\frac{1}{2}t^2}}{e^{\sqrt{2}t}} \\
    &= e^{\frac{1}{2}t^2 - \sqrt{2}t},
\end{align}
for every $t > 0$. This inequality is tightest for $t = \sqrt{2}$, as this minimizes $\frac{1}{2}t^2 - \sqrt{2}t$. Plugging in this value yields
\begin{align}
    \P{Z > \sqrt{2}} &\leq e^{\frac{1}{2}\sqrt{2}^2 - \sqrt{2}\cdot \sqrt{2}}\\
    &\leq e^{1 - 2} = 1/e.
\end{align}
\end{solution}
\end{exercise}


\subsection{BH Exercises}


\begin{exercise}
BH Exercises: 7.15,
\begin{solution}
See our solution manual.
\end{solution}
\end{exercise}


\section{Homework}
\label{sec:homework}

\subsection{Practice}
\label{sec:practice}




\subsection{BH Exercises}
\label{sec:bh-exercises-1}


\begin{exercise}
BH 7.24
\begin{solution}
See our solution manual.
\end{solution}
\end{exercise}



\section{Polleverywhere questions}
\label{sec:polev-quest}


TBD


\section{Assignment}
\label{sec:assignment}

\subfile{../assignments/bh-9-37.tex}


\opt{all-solutions-at-end}{
\Closesolutionfile{hint}
\Closesolutionfile{ans}
\clearpage
\section{Hints}
\input{hint}
\clearpage
\section{Solutions}
\input{ans}
}
\end{document}
