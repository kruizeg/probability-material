% arara: pdflatex: { shell: yes }
% arara: pythontex: {verbose: yes, rerun: always }
% arara: pdflatex: { shell: yes }

\documentclass[a4paper,11pt]{article}


\usepackage[all-solutions-at-end]{optional}

\usepackage{../common/preamble}
%\newcommand{\sectionbreak}{} %\clearpage}


\newtheorem{theorem}{Theorem}
\newtheorem{exercise}[theorem]{Ex}
\newtheorem{remark}[theorem]{Remark}

\setcounter{tocdepth}{1}
\usepackage{a4wide}
\usepackage{../common/abbreviations}

\author{Lecturers PT and PD
}
\date{\today}
\title{Probability Distributions, Week 4}

\opt{all-solutions-at-end}{
\Opensolutionfile{hint}
\Opensolutionfile{ans}
}


\begin{document}
\maketitle
\tableofcontents

\section{Lecture 7}
\label{sec:lecture-1}


\subsection{Compulory material}
\label{sec:compulory-material}

\begin{itemize}
\item 9.1: read, but skip example 9.1.6 on the two envelopes paradox.
\item BH video: 26
\end{itemize}

\begin{pycode}
from pathlib import Path

exercise_name = "bh-9.1.tex"
fname = Path("../bh_problems") / exercise_name
with fname.open("r") as fp:
    state = 0  # dump
    for line in fp.readlines():
        if line[:16] == r"\begin{exercise}":
            state = 1
        if state == 1:
            print(line.strip())
        if line[:14] == r"\end{exercise}":
            break
\end{pycode}



The next real exercise is about recursion applied to the negative hypergeometric distribution. But to get in the mood, here is short fun question on how to use recursion.

\begin{exercise}
We have a chocolate bar consisting of $n$ small squares.
The bar is in any shape you like, square, rectangular, whatever.
What is the number of times you have to break the bar such that you end up with the $n$ single pieces?
%Recursive arguments are very powerful, and I often find them very insightful (and I also like this type of reasoning).
\begin{solution}
  Write $T(n)$ for the number of times you have to break the bar when there are $n$ squares.
  Clearly, $T(1) =0$.
  In general, suppose $n=m+k$, $1 \leq k < n$, then $T(n) = T(m) + T(k) + 1$, because we need $T(k)$ to break the part with $k$ pieces, and $T(m)$ for the part with $m$ pieces, and we need 1 to break the big bar into the two parts. But then, with the \emph{boundary condition} $T(1)=0$,
  \begin{align*}
    T(2) &= T(1) + T(1) + 1 = 1, \\
T(3) &= T(1) + T(2) + 1 = 1 + 1 = 2,
  \end{align*}
and so on. So we guess that $T(n)=n-1$ for any $n$. Let's check. It certainly holds for $n=1$.  Generally,
\begin{align*}
n-1 = T(n) = T(k) + T(m) + 1 = k-1 + m-1 + 1 = k+m-1 = n-1.
\end{align*}
\end{solution}
\end{exercise}

\begin{exercise}
Use recursion to find the expected number $X$ of black balls drawn without replacement at random from an urn containing $w\geq 1$ white balls and $b$ black balls before we draw 1 white ball.
In other words, I ask to use recursion to compute $\E X$ for $X$ a negative hypergeometric distribution with parameters $w,b, r=1$ and show that
\begin{align}
  \label{eq:727}
\E X = \frac{b}{w+1}
\end{align}
\begin{hint}
For ease, write $N(w,b)= \E X$ for an  urn with $w\geq r = 1$ white balls and $b$ black balls. Then explain that
\begin{align}
N(w, 0) &= 0\quad \text{ for all $w$},\\
  N(w,b) &= \frac{b}{w+b} (1+N(w, b-1)).
\end{align}
Then show that this implies that $N(w,b) = b/ (w+1)$.
\end{hint}
\begin{solution}
If there are no black balls left, then we cannot pick a black ball which implies that $N(w, 0) = 0$.
As soon as we pick a white ball, we have to stop.

When $w, b \geq 1$ two things can happen.
Suppose we pick a white ball, then we stop right away and we cannot increase the number of black balls picked.
Suppose we pick a black ball, which happens with probability $b/(b+w)$, we obtain one black ball, and we continue with one black ball less.
Hence, with LOTE,
\begin{align}
  N(w,b) = \frac{b}{w+b} (1+N(w, b-1)).
\end{align}
You should realize that here we use conditioning; we condition on the color of the ball picked.

Now we need a real tiny bit of inspiration.
Let's \emph{guess} that $N(w,b)=\alpha b$ for some $\alpha>0$.
Guessing is always allowed; if it works, we are done (since the recursion, together with the boundary condition, has a unique solution: just repeatedly applying it allows us to calculate all values).
In general, you can make any guess whatsoever, but mind that only the good guesses work. So, trying the form $N(w,b) = \alpha b$, we find that
\begin{equation*}
\alpha b = \frac{b}{w+b} (1+ \alpha (b-1)) \stackrel{\textrm{\tiny algebra}}{\implies} \alpha = \frac{1}{w+1} \implies  N(w, b) = \frac{b }{w+1}.
\end{equation*}
This is consistent with the boundary condition $N(w,0)=0$. Hence, we can move the case $b=0$ to the case with $b=1$, and so on, thereby proving the validity of the formula in general.
\end{solution}
\end{exercise}

\begin{exercise}
Extend the previous exercise to cope with the case $r\geq 2$.
For this, write $N_{r}(w,b)$ for an urn with $w$ white balls and $b$ black balls, and $r$ white balls to go.
\begin{hint}
Explain that $N_{0}(w,b) = 0$ and
\begin{align}
  N_r(w,b) &= \frac{w}{w+b} N_{r-1}(w-1, b) +  \frac{b}{w+b} (1+N_{r}(w, b-1)).
\end{align}
Then show that this implies that $N_{r}(w,b) = r b/ (w+1)$.
\end{hint}
\begin{solution}
The recursion follows right away by noticing that we pick a white ball with probability $w/(w+b)$, but then we remove one white ball \emph{and} we have one white ball less to go.
With probability $b/(w+b)$ we pick a black ball, in which case the number of black balls drawn increases by one, but there is one black ball less in the urn.

The boundary condition is also clear: $N_{0}(w,b)=0$ because we stop.
Note that this is consistent with the previous exercise.

If $r=2$, we use from the previous exercise that
\begin{equation*}
N_{r-1}(w-1, b) = N_{1}(w-1, b) = \frac{b }{w-1 + 1} = \frac{b }{w}.
\end{equation*}
Using this in the recursion,
\begin{align*}
  N_2(w,b)
&= \frac{w}{w+b} N_{1}(w-1, b) +  \frac{b}{w+b} (1+N_{2}(w, b-1)) \\
&= \frac{w}{w+b} \frac{b }{w} +  \frac{b}{w+b} (1+N_{2}(w, b-1)) \\
&=  \frac{b}{w+b} (2+N_{2}(w, b-1)).
\end{align*}
But this has the same form as~\cref{eq:727}, except that the 1 has been replaced by a 2. But, now we are dealing with the case $r=2$ instead of $r=1$. Hence, by the same token,
\begin{equation*}
  N_2(w,b) = 2 \frac{b}{w+1}
\end{equation*}
Knowing the result for $r=2$, we fill this for $r=3$, and so on, to get the general result.

What an elegant procedure;  we pull ourselves out of the swamp, just like Baron Munchhausen.
\end{solution}
\end{exercise}

\section{Lecture 8}


\subsection{Compulory material}
\label{sec:compulory-material}

\begin{itemize}
\item 9.2: read
\item BH video: 27
\end{itemize}


\begin{exercise}
We draw, with replacement, balls, numbered 1 to $N$, from an urn.
Find a recursion to compute the expected number $\E T$ of draws necessary to see all balls.
\begin{solution}
Write $T_{n}$ for expected time to finish given that we have seen $n$ different balls. Thus, $\E T = T_{0}$.
Then for $n< N$ and using conditioning and LOTE
\begin{align}
T_n
&= 1
+T_{n+1} \P{\text{draw a new ball}}
+T_{n} \P{\text{draw an old ball}}  \\
&= 1
+T_{n+1} \frac{N-n}{N}
+T_{n} \frac{n}{N} \\
&\implies\\
T_n \frac{N-n}{N} &= 1 +T_{n+1} \frac{N-n}{N} \\
&\implies \\
T_n &= \frac{N}{N-n}  +T_{n+1}.
\end{align}
How about the boundary conditions, i.e., what is $T_{n}$ for $n=N$?
That's obvious: $T_{N} = 0$. Hence,
\begin{equation}
T_{N-1} = \frac{N}{N-(N-1)} + T_{N} = N,
\end{equation}
and so on.

If you are interested,  push the result a bit further: approximate $T_{0}$ for $N\gg 0$.
\end{solution}
\end{exercise}


\begin{exercise}
Write code to compute $\E T$ for $N=45$.
\begin{solution}
Here is the python code.

\begin{pyblock}
N = 45


def T(n):
    if n >= N:
        return 0
    return N / (N - n) + T(n + 1)

# print(T(44))
# print(T(0))
\end{pyblock}
\begin{minted}{R}
N = 45

bigT = function(n) {
  if(n >= N) {
    return(0)
  }
  return(N / (N - n) + bigT(n + 1))
}

#print(bigT(44))
#print(bigT(0))
\end{minted}
Here are the results $T(44) = \py{T(44)}$ (check!) and $T(0) = \py{T(0)}$.
\end{solution}
\end{exercise}

\begin{exercise}
We draw, with replacement, balls, numbered 1 to $N$, from an urn, but 6 at a time (not just one as in the previous exercise).
Find a recursion to compute the expected number $\E T$ of draws necessary to see all balls.
\begin{solution}
  Write $T_{n}$ for expected time to finish given that we have seen $n$ different balls.
  Take 6 balls.
  If we would know the number $k$ of new balls drawn, then $T_{n} = 1 + T_{n+k}$.
  What is the probability to draw $k$ new balls  out of the 6 we pick?
  This must be
\begin{equation}
\label{eq:930}
{n \choose 6-k}{N-n \choose k}\big/{N \choose 6}.
\end{equation}
Therefore, by LOTE, and when $N-n\geq 6$
\begin{align}
T_{n} = 1 + \sum_{k=0}^{6} \frac{{n \choose 6-k}{N-n \choose k}}{{N \choose 6}} T_{n+k}.
\end{align}
This formula is not ok when are just 2 new balls as in that case we cannot pick $k=6$ new balls. In general, we can pick $k=\min\{6, N-n\}$ new balls. Hence,
\begin{align}
T_{n}
&=
1 + \sum_{k=0}^{\min\{6, N-n\}} \frac{{n \choose 6-k}{N-n \choose k}}{{N \choose 6}} T_{n+k} \\
&\implies \\
T_{n} - \frac{{n \choose 6}{N-n \choose 0}}{{N \choose 6}} T_{n}
&=1 + \sum_{k=1}^{\min\{6, N-n\}} \frac{{n \choose 6-k}{N-n \choose k}}{{N \choose 6}} T_{n+k} \\
&\implies \\
T_{n}\left( {N\choose 6}  -{n \choose 6} \right)
&={N\choose 6} + \sum_{k=1}^{\min\{6, N-n\}}{n \choose 6-k}{N-n \choose k} T_{n+k} \\
\end{align}
And this is the final result.

\end{solution}
\end{exercise}

\begin{exercise}
For the previous exercise, compute $\E T$ for $N=45$.
\begin{solution}
  When we draw $m=6$ balls at a time, we should take care how we compute the recursion.

  To see this, consider for ease the case in which we draw two balls at at time.  Now I want to know how often each function gets called in the computation. For this, I write a simple helper function $S(n, N)$ that returns $1$ for being called plus $S(n+1, N) + S(n+2, N)$. Like this, I can find out how often the functions are called in the recursion.

\begin{pyblock}
def S(n, N):
    if n >= N:
        return 0
    return 1 + S(n + 1, N) + S(n + 2, N)


for N in range(1, 25):
    print(S(0, N))
\end{pyblock}
\begin{minted}{R}
S = function(n, N) {
  if(n >= N) {
    return(0)
  }
  return(1 + S(n + 1, N) + S(n + 2, N))
}

for (N in 1:24) {
  print(S(0, N))
}
\end{minted}
The result is this: \printpythontex.
Apparently, the number of functions called grows very rapidly.
When we want to compute for 6 balls and $N=45$ the situation must be much, much worse.

The way out is to \emph{store} intermediate results rather than computing them time and again.
For this we use the concept \emph{memoization}.
You should memorize this concept when dealing with the computation of recursions.
In short, this means that we check whether we computed the value of a function earlier.
If so, get the value from memory, otherwise, do the computation, and store it for later purposes.

Let's check what happens if we print out how often the function gets called when using memoization.
\begin{pyblock}
from functools import lru_cache

@lru_cache   # This realizes the memoization.
def S(n, N):
    if n >= N:
        return 0
    print(f"Called for n = {n}\n")
    return 1 + S(n + 1, N) + S(n + 2, N)


S(0, 5)
\end{pyblock}
\begin{minted}{R}
library("memoise")

S = function(n, N) {
  if(n >= N) {
    return(0)
  }
  cat("Called for n =", n, "\n")
  return(1 + S(n + 1, N) + S(n + 2, N))
}
S = memoise(S)

S(0, 5)
\end{minted}
Now the result is this:
\begin{center}
\printpythontex
\end{center}
This is much better, for each $n$ the function in the recursion gets called just once.

To convince you that memoization really works, let's remove the memoization.
\begin{pyblock}
def S(n, N):
    if n >= N:
        return 0
    print(f"Called for n = {n}\n")
    return 1 + S(n + 1, N) + S(n + 2, N)


S(0, 5)
\end{pyblock}
\begin{minted}{R}
S = function(n, N) {
  if(n >= N) {
    return(0)
  }
  cat("Called for n =", n, "\n")
  return(1 + S(n + 1, N) + S(n + 2, N))
}

S(0, 5)
\end{minted}
This is the result:
\begin{center}
\printpythontex
\end{center}
You see how often the function is called with $n=4$?



For the course you don't have to look up on the web how memoization is implemented, but I advice you to read about.
Understanding such ideas makes you much better at programming, which is important in view of the fact that many of you will have to use computers a lot in your profession careers.
Besides this, memoization lies at the heart of how spreadsheet programs such as excel work.

Now we know that we have to use memoization, we can turn to the balls and urn problem.
I build a general function $T(n,m)$ for the situation in which we have seen $n$ balls and we pick $m$ balls at a time.
Like this, I can check with the earlier case in which we picked just one ball by computing $T(n, 1)$.
(Hopefully you learn from this that you should also test your code.)


%\begin{pyblock}[][numbers=left]
\begin{pyblock}
from math import comb
from functools import lru_cache

N = 45


@lru_cache
def T(n, m):
    if n >= N:
        return 0
    res = comb(N, m)
    for k in range(1, min(m, N - n) + 1):
        P = comb(n, m - k)
        P *= comb(N - n, k)
        res += P * T(n + k, m)
    return res / (comb(N, m) - comb(n, m))

\end{pyblock}
\begin{minted}{R}
library("memoise")

N = 45

bigT = function(n, m) {
  if(n >= N) {
    return(0)
  }
  res = choose(N, m)
  for (k in 1:min(m, N - n)) {
    P = choose(n, m - k)
    P = P * choose(N - n, k)
    res = res + P * bigT(n + k, m)
  }
  return(res / (choose(N, m) - choose(n, m)))
}
bigT = memoise(bigT)

bigT(0, 1)
bigT(0, 6)
\end{minted}
Here is the check: $T(0, 1) = \py{T(0, 1)}$, which is the same as our earlier computation.

Next, $T(0, 6) = \py{T(0,6)}$.
This is more than $6$ as fast, i.e., $6\times T(0, 6) = \py{6*T(0, 6)} < T(0,1)$.
Why is that?
Well, the 6 balls we pick from the urn are guaranteed to be drawn without replacement.


Can you solve the following extension? Suppose for each time you draw a set of balls, you get the value (in Euro's say) of the numbers on the balls, and then you put the balls back in the urn.
The game stops when you have seen all balls at least once. What is your expected gain?

\end{solution}
\end{exercise}


\begin{pycode}
from pathlib import Path

exercise_name = "bh-9.25.tex"
fname = Path("../bh_problems") / exercise_name
with fname.open("r") as fp:
    state = 0  # dump
    for line in fp.readlines():
        if line[:16] == r"\begin{exercise}":
            state = 1
        if state == 1:
            print(line.strip())
        if line[:14] == r"\end{exercise}":
            break
\end{pycode}


\section{Tutorial}
\label{sec:tutorial}


\subsection{Practice}

\begin{exercise}
Let $X$ be a continuous random variable with a pdf
\begin{align}
    f_X(x) = \begin{cases}
    c, &\text{if } 0 \leq x \leq 4, \\
    0, &\text{otherwise}.
    \end{cases}
\end{align}
\begin{enumerate}
    \item What is the value of $c$?
    \item What is the distribution of $X$?
    \item Do we need to know the value of $c$ to determine the distribution of $X$?
\end{enumerate}

\begin{solution}
We need that the pdf $f_X$ integrates to one. Hence, we need
\begin{align}
    \int_{-\infty}^\infty f_X(x) dx &= 1 \qquad \iff \\
    \int_{0}^4 c dx &= 1 \qquad \iff \\
    4c &= 1 \qquad \iff \\
    c &= 1/4.
\end{align}
Clearly, $X$ is uniformly distributed on $[0,4]$. In fact, we do not need to know the value of $c$ to determine this. It is sufficient to know that the pdf of $X$ is constant on the interval $[0,4]$.
\end{solution}
\end{exercise}

\begin{exercise}
Let $X$ be a continuous random variable with a pdf
\begin{align}
    f_X(x) = c \cdot e^{-\frac{(x - 4)^2}{8}}, \quad x \in \R.
\end{align}
\begin{enumerate}
    \item What is the value of $c$?
    \item What is the distribution of $X$?
    \item Do we need to know the value of $c$ to determine the distribution of $X$?
\end{enumerate}
\begin{solution}
We need that the pdf $f_X$ integrates to one. Hence, we need
\begin{align}
    \int_{-\infty}^\infty f_X(x) dx &= 1 \qquad \iff \\
    \int_{-\infty}^\infty c \cdot e^{-\frac{(x - 4)^2}{8}} dx &= 1 \qquad \iff \\
    c \cdot \sqrt{2\pi}\cdot 2 \int_{-\infty}^\infty  \frac{1}{\sqrt{2\pi}\cdot 2} e^{-\frac{1}{2}\frac{(x - 4)^2}{2^2}} dx &= 1 \qquad \iff \\
    c \cdot \sqrt{2\pi}\cdot 2 \cdot 1 &= 1 \qquad \iff \\
    c &= \frac{1}{\sqrt{2\pi}\cdot 2}.
\end{align}
Clearly, $X$ is $N(4,2)$ distributed. In fact, we do not need to know the value of $c$ to determine this. It is sufficient to observe the structure of the pdf as a function of $x$.
\end{solution}
\end{exercise}


\subsection{BH Exercises}


\begin{pycode}
from pathlib import Path

exercise_name = "bh-9.32.tex"
fname = Path("../bh_problems") / exercise_name
with fname.open("r") as fp:
    state = 0  # dump
    for line in fp.readlines():
        if line[:16] == r"\begin{exercise}":
            state = 1
        if state == 1:
            print(line.strip())
        if line[:14] == r"\end{exercise}":
            break
\end{pycode}

\section{Homework}

\label{sec:homework}

\subsection{Practice}
\label{sec:practice}



\subsection{BH Exercises}
\label{sec:bh-exercises-1}

\begin{pycode}
from pathlib import Path

exercise_name = "bh-9.28.tex"
fname = Path("../bh_problems") / exercise_name
with fname.open("r") as fp:
    state = 0  # dump
    for line in fp.readlines():
        if line[:16] == r"\begin{exercise}":
            state = 1
        if state == 1:
            print(line.strip())
        if line[:14] == r"\end{exercise}":
            break
\end{pycode}



\section{Poleverywhere questions}
\label{sec:polev-quest}


TBD


\section{Assignment}
\label{sec:assignment}

For the handwritten solution, take the following exercise.

\begin{pycode}
from pathlib import Path

exercise_name = "bh-8.13.tex"
fname = Path("../bh_problems") / exercise_name
with fname.open("r") as fp:
    state = 0  # dump
    for line in fp.readlines():
        if line[:16] == r"\begin{exercise}":
            state = 1
        if state == 1:
            print(line.strip())
        if line[:14] == r"\end{exercise}":
            break
\end{pycode}


\subfile{../assignments/bh-8-18.tex}



\opt{all-solutions-at-end}{
\Closesolutionfile{hint}
\Closesolutionfile{ans}
\clearpage
\section{Hints}
\input{hint}
\clearpage
\section{Solutions}
\input{ans}
}

\end{document}
