% arara: pdflatex: { shell: yes }
% arara: pythontex: {verbose: yes, rerun: always }
% arara: pdflatex: { shell: yes }

\documentclass[notitlepage,a4paper,11pt]{article}

\usepackage[no-solutions]{optional}
\usepackage{../common/preamble}
\renewcommand\sectionbreak{}
\usepackage{a4wide}
\author{Lecturers PT and PD}
\date{\today}
\setlength{\parindent}{0pt}


\title{Probability Distributions, Week 5}

\begin{document}
\maketitle
\toccontents

\section{Lecture 9}
\label{sec:lecture-1}

\subsection{Compulory material}
\label{sec:compulory-material}


\begin{itemize}
\item 8.2: read, but do convolutions with conditioning
\item 9.3: read
\item 9.4: skip
\item BH video: 27
\end{itemize}


\begin{pycode}
from pathlib import Path

exercise_name = "bh-8.23.tex"
fname = Path("../bh_problems") / exercise_name
with fname.open("r") as fp:
    state = 0  # dump
    for line in fp.readlines():
        if line[:16] == r"\begin{exercise}":
            state = 1
        if state == 1:
            print(line.strip())
        if line[:14] == r"\end{exercise}":
            break
\end{pycode}


\begin{exercise}
The lifetime $X$ of a machine is $\Exp{\lambda}$.

a. Compute $\E{X \given X\leq \tau}$ where $\tau$ is some positive constant.

b. As a \emph{check} on the result of a., use  Adam's law and LOTP to show that $\E{X} = \E{\E{X\given Y}} = 1/\lambda$, where $Y=\1{X\leq \tau}$.
\begin{solution}
Since $X\sim \Exp{\lambda}$,
\begin{equation*}
\P{X\leq t | X\leq \tau} = \frac{\P{X\leq \min\{t, \tau\}}}{\P{X\leq \tau}} = \frac{1-e^{-\lambda\min\{t, \tau\}}}{1-e^{-\lambda \tau}}.
\end{equation*}
 Defining
\begin{equation*}
  f(t) = \partial_{t} \P{X\leq t | X\leq \tau} = \lambda \frac{e^{-\lambda t}}{1-e^{-\lambda \tau}} \1{0\leq t \leq \tau},
\end{equation*}
we get
\begin{align}
  \label{eq:9925}
\E{X|X\leq \tau}
&= \int t f(t) \d t \\
&= \frac{1}{1-e^{-\lambda \tau}}\int_{0}^{\tau} \lambda t e^{-\lambda t} \d t.
\end{align}
Simplifying,
\begin{align}
\label{eq:926}
\int_0^{\tau} \lambda t e^{-\lambda t} \d t
&= \left.- t e^{-\lambda t}\right|_{0}^{\tau} + \int_0^{\tau} e^{-\lambda t} \d t \\
&= -\tau e^{-\lambda \tau} + \frac{e^{-\lambda \tau} -1}{-\lambda} \\
&= \frac{ 1- e^{-\lambda \tau}- \tau \lambda  e^{-\lambda \tau}}{\lambda},
\end{align}
\begin{align}
  \label{eq:925}
\E{X|X\leq \tau}
&= \frac{ 1- e^{-\lambda \tau}- \tau \lambda  e^{-\lambda \tau}}{\lambda(1-e^{-\lambda \tau})}.
\end{align}

For part 2: by memorylessness, $\E{X\given X>\tau} = \tau + 1/\lambda$. Hence,
\begin{align}
  \label{eq:94}
\E X
&= \E{\E{X\given Y}}, \quad \text{Adam's law} \\
&= \E{X\given Y=1}\P{Y=1} + \E{X\given Y=0}\P{Y=0}, \quad \text{LOTE}\\
&= \E{X\given X\leq \tau}\P{X\leq \tau} + \E{X\given X>\tau} \P{X>\tau} \\
&= \frac{ 1- e^{-\lambda \tau}- \tau \lambda  e^{-\lambda \tau}}{\lambda(1-e^{-\lambda \tau})} (1-e^{-\lambda \tau}) + (\tau+ 1/\lambda) e^{-\lambda \tau} \\
&= 1/\lambda - e^{-\lambda \tau}/\lambda - \tau e^{-\lambda \tau} + \tau e^{-\lambda T} + e^{-\lambda \tau}/\lambda \\
&=1/\lambda.
\end{align}

Here is a final point for you to think hard about: ${X\leq \tau}$ is an event, $Y=\1{X \leq \tau}$ is a r.v. Therefore the following is sloppy notation: $\E{\E{X\given X\leq \tau}}$, and when using, it is easy to get confused, because where is the random variable  on which we  should condition according to Adam's law?
\end{solution}
\end{exercise}



\section{Lecture 10}

\subsection{Compulory material}
\label{sec:compulory-material}


\begin{itemize}
\item 9.5: read but skip example 9.5.4
\item 9.6: read but skip example 9.6.3
\item BH video: -
\end{itemize}



\begin{pycode}
from pathlib import Path

exercise_name = "bh-9.50.tex"
fname = Path("../bh_problems") / exercise_name
with fname.open("r") as fp:
    state = 0  # dump
    for line in fp.readlines():
        if line[:16] == r"\begin{exercise}":
            state = 1
        if state == 1:
            print(line.strip())
        if line[:14] == r"\end{exercise}":
            break
\end{pycode}


\begin{pycode}
from pathlib import Path

exercise_name = "bh-9.52.tex"
fname = Path("../bh_problems") / exercise_name
with fname.open("r") as fp:
    state = 0  # dump
    for line in fp.readlines():
        if line[:16] == r"\begin{exercise}":
            state = 1
        if state == 1:
            print(line.strip())
        if line[:14] == r"\end{exercise}":
            break
\end{pycode}


\begin{exercise}
We have a station with two machines, one is working, the other is off.
If the first fails, the other machine takes over.
The repair time of the first machine is a constant $\tau$.
If the second machine fails before the first is repaired, the station stops working, i.e., is `down'. If the second machine does not fail before the first is repaired, the first machine takes over, and a new cycle starts (i.e., wait for the first to fail again, and so on).

Use a conditioning argument to find the expected time $\E T$ until the station is down when the lifetimes of both machines is iid $\sim\Exp{\lambda}$.
\begin{solution}
Let $X_i$ be the lifetime of machine $i$. Suppose that  the second machine does not break down before $\tau$, then
\begin{equation}
\E{T| X_{2}>\tau} = \E{X_{1} + \tau + C |X_2>\tau} = 1/\lambda + \tau + \E T,
\end{equation}
because the expected time until machine 1 fails is $1/\lambda$, then we wait until it is repaired, and then, by memoryless, a new cycle $C$ starts, which has an expected duration of $\E T$.

Next condition on the second machine breaking down before $\tau$. Then, since $X_{1}$ and $X_2$ are independent and by the previous exercise,
\begin{equation}
\E{T| X_{2}\leq \tau} = \E{X_{1} + X_{2}  |X_2\leq \tau} = \frac{1}{\lambda} + \frac{ 1- e^{-\lambda \tau}- \tau \lambda  e^{-\lambda \tau}}{\lambda(1-e^{-\lambda \tau})}.
\end{equation}

By LOTE and using $\P{X_{2}\geq \tau} = e^{-\lambda \tau}$,
\begin{align}
  \label{eq:924}
\E T
&=
\E{T| X_{2}\leq \tau} \P{X_{2}\leq \tau} + \E{T| X_{2}\geq \tau} \P{X_{2}\geq \tau} \\
&=(1-e^{-\lambda \tau})\left( \frac 1\lambda+
\frac{ 1- e^{-\lambda \tau}- \tau \lambda  e^{-\lambda \tau}}{\lambda(1-e^{-\lambda \tau})} \right)
+e^{-\lambda \tau}\left( \frac{1}{\lambda} + \tau + \E T \right) \\
&=\frac{1}{\lambda} + \frac{ 1- e^{-\lambda \tau}- \tau \lambda  e^{-\lambda \tau}}{\lambda}
+e^{-\lambda \tau} \tau + e^{-\lambda \tau}\E T \\
&\implies \\
\left( 1-e^{-\lambda \tau} \right) \E T
&=\frac{2-e^{-\lambda \tau}}{\lambda} \\
&\implies \\
\E T
&=\frac{2-e^{-\lambda \tau}}{\lambda(1-e^{-\lambda \tau})}.
\end{align}

For you to do: Check the limits $\tau=0$ and $\tau=\infty$. Is the result in accordance with your intuition?

Two challenges: Does the problem become much harder when $X_{1}$ and $X_{2}$ are still exponential, but have different failure rates $\lambda_1$ and $\lambda_{2}$? Can you extend the analysis to 3 machines, or $n$ machines?

\end{solution}
\end{exercise}


\section{Tutorial}
\label{sec:tutorial}


\subsection{Practice}

We have a wooden stick of length 100 cm that we break twice.
First, we break the stick at a random point that is uniformly distributed over the entire stick.
We keep the left end of the stick.
Then, we break the remaining stick again at a random point, uniformly distributed again, and we keep the left end again.

\begin{exercise}
What is the expected length of the stick we end up with?
\begin{solution}
Let $X \sim \Unif{0,100}$ denote the point where the first stick is broken (in cm). Let $S$ denote the point where the second stick is broken (i.e., the length of the remaining stick). Then, $S \sim \Unif{0,X}$. We want to compute $\E{Y}$. Using Adam's law we obtain
\begin{align}
    \E{S} &= \E{\E{S|X}} \\
    &= \E{X/2} \\
    &= \E{X}/2 \\
    &= 100/4 = 25.
\end{align}
Hence, we end up with a stick of expected length 25 cm.
\end{solution}
\end{exercise}

\begin{exercise}
Now we change the story slightly. Every time we break a stick, we keep the \textit{longest} part. What is the expected length of the remaining stick?
\begin{solution}
By symmetry we can assume that the point of the first break is $Y\sim\Unif{1/2, 1}$, in meters. By the same reasoning, the point of the second break, conditional on $Y=y$, is $Z|Y=y \sim \Unif{y/2, y}$. Then, $\E{Z|Y=y} = 3y/4$. Replacing $y$ by the rv $Y$, we have that $\E{Z|Y} = 3 Y/4$. With this, and Adam's law,
\begin{equation}
\E Z = \E{\E{Z|Y}} = \E{3Y/4} = 3/4\cdot \E{Y}= 3/4\cdot (1/2+1)/2 = (3/4)^{2},
\end{equation}
in meters.

\end{solution}
\end{exercise}


\begin{exercise}
As a continuation of BH.9.2.4, we now ask you to use Eve's law to compute $\V Y$, the variance of the second break point.
\begin{solution}
Recall that $\V X = 1/12$. More generally, if the stick has length $l$, then $\V X = l^{2}/12$. With this idea and using the solution of BH.9.2.4,
\begin{align}
\E{Y\given X=x} &= x/2,\\
\V{Y\given X=x}
&= \E{Y^{2}\given X=x} - (\E{Y\given X=x})^{2} \\
&= \frac{1}{x}\int_{0}^{x} y^{2} \d y - \left(\frac1x\int_{0}^{x} y \d y\right)^{2}  = x^{3}/3x - (x^{2}/2x)^{2} = x^{2}/12\\
\intertext{Replacing $x$ by $X$:}
\E{Y\given X} &= X/2,\\
\V{Y\given X} &= X^{2}/12, \\
\intertext{Adam's law:}
\E{Y\given X} &= X/2 \implies \E{Y} = \E{\E{Y\given X}} = \E{X/2} = 1/4.
\intertext{Eve's law:}
\V Y &= \E{\V{Y\given X}} + \V{\E{Y\given X}} \\
\V{Y\given X} &= X^{2}/12 \implies \E{\V{Y\given X}} = \E{X^2}/12 = \frac{1}{12} \int_0^1x^2 \d x = \frac{1}{12\cdot 3}\\
\E{Y\given X} &= X/2 \implies \V{\E{Y\given X}} = \V{X/2} = \frac{1}{4} \frac{1}{12}, \\
\V Y &= 1/12\cdot (1/3 + 1/4) = 7/144.
\end{align}
\end{solution}
\end{exercise}



\subsection{BH Exercises}


\begin{pycode}
from pathlib import Path

exercise_name = "bh-9.39.tex"
fname = Path("../bh_problems") / exercise_name
with fname.open("r") as fp:
    state = 0  # dump
    for line in fp.readlines():
        if line[:16] == r"\begin{exercise}":
            state = 1
        if state == 1:
            print(line.strip())
        if line[:14] == r"\end{exercise}":
            break
\end{pycode}


\section{Homework}
\label{sec:homework}

\subsection{Practice}
\label{sec:practice}




\begin{exercise}
We have a standard fair die and we paint red the sides with $1, 2, 3$ and blue the other sides.
Let $X$ be the color after a throw and $Y$ the number.
Use the definitions of conditional expectation, conditional variance and Adam's and Eve's law to compute $\E Y$ and $\V Y$.
Compare that to the results you would obtain from using the standard methods; you should get the same result.

This exercises is handy to check before you solve BH.9.1.


BTW, it's easy to make some variations on this exercise by painting other combinations of sides, e.g., only 1 and 2 are red.
\begin{solution}
The first step is easy:
\begin{align}
\E{Y\given X=r} &= \sum_{y} y \P{Y=y\given X=r} = \sum_{y} y \frac{\P{Y=y,  X=r}}{\P{X=r}} = \frac{(1+2+3)/6}{1/2}  = 2, \\
\E{Y\given X=b} &= \frac{(4 + 5+ 6)/6}{1/2}  = 5.
\end{align}

Recall that in the stick breaking exercise, we first computed $\E{Y\given X = x} = x/2$, and then we replaced $x$ by $X$ to get $\E{Y\given X} = X/2$. In the present case (with the die), where is the $x$ at the RHS? In other words, how to turn the above into a function $g(x)$ so that we can replace $x$ by $X$ to get a rv?


Here is the solution. Define $g$ as
\begin{equation}
  \label{eq:111}
  g(x) =
  \begin{cases}
    \E{Y\given X=r}, & \text{if } x = r \\
    \E{Y\given X=b}, & \text{if } x = b.
  \end{cases}
\end{equation}
But this is equal to
\begin{equation}
  \label{eq:112}
  g(x) = \E{Y\given X=r} \1{x=r} + \E{Y\given X=b} \1{x=b}.
\end{equation}

With this insight:
\begin{align*}
\E{Y\given x} &=  g(x) = \E{Y\given X=r} \1{x=r} + \E{Y\given X=b} \1{x=b}, \\
\E{Y\given X} &=  g(X) = \E{Y\given X=r} \1{X=r} + \E{Y\given X=b} \1{X=b}.
\end{align*}
Here, the indicators are the rvs!

And now,
\begin{align}
\E Y &= \E{\E{Y\given X}} = \E{g(X)} \\
&=\E{\E{Y\given X=r} \1{X=r} + \E{Y\given X=b} \1{X=b}}\\
&=\E{\E{Y\given X=r} \1{X=r}} + \E{\E{Y\given X=b} \1{X=b}}\\
&=\E{Y\given X=r} \E{\1{X=r}} + \E{Y\given X=b} \E{\1{X=b}}\\
&=\E{Y\given X=r} \P{X=r} + \E{Y\given X=b} \P{X=b}\\
&=2 \frac{1}{2}+ 5 \frac{1}{2}.
\end{align}

In view of this, let's consider LOTE again. Observing that
\begin{align}
A_{1} = \{s \in S : \textrm{paint of $s$ is red}\} = \{1, 2,3 \} \\
A_{2} = \{s \in S : \textrm{paint of $s$ is blue}\} = \{4, 5, 6\},
\end{align}
we have
\begin{align}
  \E{Y\given X=r} &= \E{Y\given A_{1}}, \\
  \E{Y\given X=b} &= \E{Y\given A_{2}} \\
  g(x) &= \E{Y\given X=r} \1{x=r} + \E{Y\given X=b} \1{x=b} \\
   &= \E{Y\given A_{1}} \1{x\in A_{1}} + \E{Y\given A_{2}} \1{x=A_{2}} \\
 &= \sum_{i=1,2}  \E{Y\given A_{i}} \1{x\in A_{i}}
\end{align}
The last is simple to generalize!
\begin{align}
  g(x)  &= \sum_{i}  \E{Y\given A_{i}} \1{x\in A_{i}}  \\
  g(X)  &= \sum_{i}  \E{Y\given A_{i}} \1{X\in A_{i}}.
\end{align}
And now we define $\E{Y\given X} = g(X)$. Remember these general equations!


Now observe that Adam's law and LOTE are the same for discrete rvs:
\begin{align*}
\E{Y}
&= \E{\E{Y\given X}} \\
&= \E{g(X)} \\
&=\E{\sum_{i} \E{Y\given A_{i}} \1{X\in A_{i}}} \\
&=\sum_{i}  \E{Y\given A_{i}} \E{\1{X\in A_{i}}} \\
&=\sum_{i}  \E{Y\given A_{i}} \P{A_{i}}.
\end{align*}
\end{solution}
\end{exercise}



\subsection{BH Exercises}
\label{sec:bh-exercises-1}

\begin{pycode}
from pathlib import Path

exercise_name = "bh-9.1.tex"
fname = Path("../bh_problems") / exercise_name
with fname.open("r") as fp:
    state = 0  # dump
    for line in fp.readlines():
        if line[:16] == r"\begin{exercise}":
            state = 1
        if state == 1:
            print(line.strip())
        if line[:14] == r"\end{exercise}":
            break
\end{pycode}


\begin{pycode}
from pathlib import Path

exercise_name = "bh-9.57.tex"
fname = Path("../bh_problems") / exercise_name
with fname.open("r") as fp:
    state = 0  # dump
    for line in fp.readlines():
        if line[:16] == r"\begin{exercise}":
            state = 1
        if state == 1:
            print(line.strip())
        if line[:14] == r"\end{exercise}":
            break
\end{pycode}


\begin{pycode}
from pathlib import Path

exercise_name = "bh-9.58.tex"
fname = Path("../bh_problems") / exercise_name
with fname.open("r") as fp:
    state = 0  # dump
    for line in fp.readlines():
        if line[:16] == r"\begin{exercise}":
            state = 1
        if state == 1:
            print(line.strip())
        if line[:14] == r"\end{exercise}":
            break
\end{pycode}



\section{Poleverywhere questions}
\label{sec:polev-quest}


TBD


\section{Assignment}
\label{sec:assignment}

%\subfile{../assignments/bh-7-1.tex}

\subfile{../assignments/bh-9-6-1.tex}



\opt{all-solutions-at-end}{
\Closesolutionfile{hint}
\Closesolutionfile{ans}
\clearpage
\section{Hints}
\input{hint}
\clearpage
\section{Solutions}
\input{ans}
}
\end{document}
