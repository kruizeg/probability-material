\documentclass[poll_tutorial_format]{subfiles}
\begin{document}
\maketitle
\setcounter{section}{6}
\section{PD Week 1}

\subsection{Set things up}
\label{sec:set-things-up}



\setcounter{theorem}{-1}
\begin{exercise}
Claim: you should help your neighbors to set up their polleverywhere app.

\vspace{0.5cm}\noindent  Which of the following options applies?
\begin{enumerate}
\item The claim is correct.
\item The claim is false.
\end{enumerate}
\end{exercise}

\subsection{Real questions}
\label{sec:start-real-questions pt week 5}

\begin{exercise}
$X$ is a rv, $X  \in \R$. Claim: $\supp{X} = [c, \infty) \iff \P{X\leq c} = 0$.

\vspace{0.5cm}\noindent  Which of the following options applies?
\begin{enumerate}
\item The claim is correct.
\item The claim is false.
\end{enumerate}
% \begin{solution}
% No $X=c$ is possible. Moreover, $X\sim \Unif{[c, c+1]}$ has support $[c, c+1]$ which is not $[c, \infty)$.
% \end{solution}
\end{exercise}

\begin{exercise}
Write $m$ for the median $\E X$ of the rv $X$. Claim, the following definition correct:
\begin{equation}
\label{eq:1}
\V X := \E{X^2} - m^{2}?
\end{equation}
\vspace{0.5cm}\noindent  Which of the following options applies?
\begin{enumerate}
\item The definition correct.
\item The definition is false.
\end{enumerate}
% \begin{solution}
% No, $\E X$ need not be equal to the median, moreover the \emph{definition} of the variance involves the mean, not the median.
% \end{solution}
\end{exercise}

\begin{exercise}
Suppose $X$ a real-valued rv with $\supp X = [0, c]$. Claim:
\begin{equation}
\label{eq:3}
\V X = \E{X^2} - \mu^2 \leq c \E X - (\E X)^2 = (c-\E X)\E X?
\end{equation}

\vspace{0.5cm}\noindent  Which of the following options applies?
\begin{enumerate}
\item The claim is correct.
\item The claim is false.
\end{enumerate}
% \begin{solution}
% Yes.
% \end{solution}
\end{exercise}


\begin{exercise}
$X\sim \Geo{p}$. Take $s$ such that $e^{s}q < 1$.
\begin{align}
\label{eq:5}
M_{X}(s)
  &\stackrel{1}= \E{e^{sX}} \stackrel{2}= \sum_{k=1}^{\infty} p q^{k} e^{sk} \\
  &\stackrel{3}= p \sum_{k=1}^{\infty} (e^{s}q)^{k}
  \stackrel{4}= p (\sum_{k=0}^{\infty} (e^{s}q)^{k}-1) \\
&  \stackrel{5}= p /(1+e^{s}q)  - p.
\end{align}

\vspace{0.5cm}\noindent  Which of the following options applies:
\begin{enumerate}
\item all steps are correct.
\item step 2 is incorrect.
\item step 3 is incorrect.
\item step 4 is incorrect.
\item step 5 is incorrect.
\item More than one step is incorrect.
\end{enumerate}
% \begin{solution}
% Steps 2 and 5 are incorrect. Step 2: start with $k=0$, step 5: the plus should be a minus.
% \end{solution}
\end{exercise}

\begin{exercise}
Let $X$ and $Y$ be independent rvs. Claim: $F_{X+Y}(x, y) = F_X(x) + F_{Y}(y)$.

\vspace{0.5cm}\noindent  Which of the following options applies?
\begin{enumerate}
\item The claim is correct.
\item The claim is false.
\end{enumerate}
% \begin{solution}
% No. We should write $F_{X,Y}}$ rather than $F_{X+Y}$ , and since the rvs are independent, consider the product of the CDFs, not the sum.
% \end{solution}
\end{exercise}

\end{document}
