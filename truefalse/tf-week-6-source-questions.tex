\documentclass[tf-tutorial-all.tex]{subfiles}
\begin{document}


\begin{truefalse}
Eve's law says that $V(X) = \E{\V{X|N}} + \V{\E{X|N}}$.

Claim: $\E{\V{X|N}}$ is called the \emph{in-between} group variation.

\begin{solution}
False.
\end{solution}
\end{truefalse}


\begin{truefalse}
Eve's law says that $V(X) = \E{\V{X|N}} + \V{\E{X|N}}$.

Claim: $\V{\E{X|N}}$ is called the \emph{explained variance}.

\begin{solution}
True.
\end{solution}
\end{truefalse}



\begin{truefalse}
Write $g(X) = \E{Y|X}$ for two rvs $X, Y$.
Claim: This derivation is correct:
\begin{equation}
\label{eq:7}
\V{\E{Y|X}} = \E{(g(X))^2} - (\E{g(X)})^{2} = \E{(g(X))^2} - (\E{Y})^{2}.
\end{equation}

\begin{solution}
True.
\end{solution}
\end{truefalse}

\begin{truefalse}
Claim: The inequality of Cauchy-Schwarz says that $\E{(XY)^{2}} \geq \E X \E Y$.

\begin{solution}
False.
\end{solution}
\end{truefalse}


\begin{truefalse}
Claim: This is correct: $\E{X} \leq \E{X\1{X\geq 0}}$.
\begin{solution}
True.
\end{solution}
\end{truefalse}



\begin{truefalse}
Let $g$ be a function that is concave and convex at the same time, and such that $g(0) = 0$.
Claim: by Jensen's inequality: $\E{g(X)} = g(\E{X})$.

\begin{solution}
True.
\end{solution}
\end{truefalse}


\begin{truefalse}
Claim: The following reasoning is correct. For any $a\geq 0$,
\begin{equation}
\label{eq:8}
\E{|X|} \geq \E{a \1{|X|\geq a}} = a \P{|X| \geq a}.
\end{equation}

\begin{solution}
It's true.
\end{solution}
\end{truefalse}

\begin{truefalse}
The set $\{X_i : i = 1, 2, \ldots\}$ forms a set of iid rvs such that $X_i\in \{0, 1\}$ and $\P{X_i=1} = 1/2$ for all $i$.
Take $A=\{\lim_{n\to\infty} n^{-1}\sum_{i=1}^{n}X_i = 1\}$.

Claim: The strong law of large numbers implies that $A=\varnothing$.

\begin{solution}
It's false.
\end{solution}
\end{truefalse}

\begin{truefalse}
The set $\{X_i : i = 1, 2, \ldots\}$ forms a set of iid rvs such that $X_i\in \{0, 1\}$ and $\P{X_i=1} = 1/2$ for all $i$.
Take $A=\{\lim_{n\to\infty} n^{-1}\sum_{i=1}^{n}X_i = 1\}$.

Claim: The strong law of large numbers says that $\P{A} = 0$.

\begin{solution}
It's true.
\end{solution}
\end{truefalse}

\begin{truefalse}
The set $\{X_i : i = 1, 2, \ldots\}$ forms a set of iid rvs.
Claim: The weak law of large numbers states that:
\begin{equation}
\label{eq:9}
\forall \delta, \epsilon > 0: \exists m > 0 : \forall n > m: \P{|\bar X_{n}-\mu| > \epsilon} < \delta,
\end{equation}
where $\mu=\E{X_{i}}$ and $\bar X_n = n^{-1}\sum_{i=1}^{n}X_{i}$.
\begin{solution}
True.
\end{solution}
\end{truefalse}

\begin{truefalse}
Let $X_i \sim \Exp{\lambda}$ with $Y_i = 2X_i$, $i=1,2,\dots$ Claim: $\P{\bar{X}_n\bar{Y}_n \xrightarrow{} \frac{2}{\lambda^2}}=1$.

\begin{solution}
True.
\end{solution}
\end{truefalse}

\end{document}
