\documentclass[tf-tutorial-all.tex]{subfiles}
\begin{document}



\begin{truefalse}
Suppose the rv $X$ has PDF $f_{X}(x) = Ax^{-s} \1{x\geq 1}$ for $s\in (1, 2)$ where $A$ is the normalization constant.
Claim: $\E X = \infty$.
\begin{solution}
True.
\end{solution}
\end{truefalse}

\begin{truefalse}
Claim: LOTP on a discrete sample space states that $P(B) = \sum_{i=1}^{n} \P{B|A_{i}} \P{A_{i}}$, where $\{A_{i}\}$ is a set of non-overlapping subsets of the sample space.
\begin{solution}
It's false in general, because it's not given that the sets cover the state space.
\end{solution}
\end{truefalse}


\begin{truefalse}
Let $X$ be a rvs on $\R$ and $g$ a function from $\R^{2}$ to $\R$.
Claim, $g(X)$ is a rv.
\begin{solution}
False. $g$ needs to two arguments as it maps $\R^{2}$ to $\R$.
\end{solution}
\end{truefalse}

% \begin{truefalse}
% Let $X$ be a rvs on $\R$ and $g: \R \to \R, g(x) \equiv 0$. Claim: $\E{\1{g(X) = 0}} = 1$.

% \vspace{0.5cm}\noindent  Which of the following options applies?
% \begin{enumerate}
% \item The claim is correct.
% \item The claim is false.
% \end{enumerate}
% % \begin{solution}
% % correct. Apply the fundamental bridge.
% % \end{solution}
% \end{truefalse}



\begin{truefalse}
Let $X\sim \FS{p}$ with $p\in (0, 1)$, $q=1-p$. Claim: $\E X = 1+q\E X \implies \E X = 1/p$.
\begin{solution}
It's true.
\end{solution}
\end{truefalse}


\begin{truefalse}
Claim: according to 2D LOTUS: if $g$ is a function such that $g(x, y) \in \R $, and $X, Y$ two real-valued rvs, then
\begin{equation}
\E{g(X,Y)} = \int_{-\infty}^{\infty} \int_{-\infty} ^{\infty} g(x,y) f_{X}(x)f_Y(y) \d x \d y.
\end{equation}
\begin{solution}
It's false. $X$ and $Y$ need not be independent, as is suggested here.
\end{solution}
\end{truefalse}



\begin{truefalse}
Take $g(x, X) = \1{X\leq x}$ for some rv $X$ with CDF $F_{X}$. Claim: $\E{g(y, X)} = F_{X}(y)$.
\begin{solution}
It's true.
\end{solution}
\end{truefalse}




\begin{truefalse}
Let $L = \min\{X_{i} : i =1, \ldots, n\}$ with $\{X_{i}\}$ a set of iid rvs.
Claim:
\begin{equation}
\P{L\leq x} = (\P{X_{1} \leq x})^{n}.
\end{equation}
\begin{solution}
It's false. This holds for the maximum, or reverse the direction of the inequalities with respect to $x$.
\end{solution}
\end{truefalse}

\begin{truefalse}
For two continuous rvs $X, Y$ with joint distribution $F_{X,Y}(x,y)$. Write $\partial_{x}$ for $\partial/ \partial_{x}$.
Claim: $f_{X}(x) = \partial_{x} F_{X,Y}(x,y)$.
\begin{solution}
It's false. Why is this claim nonsense?
\end{solution}
\end{truefalse}

\begin{truefalse}
Claim: for two continuous rvs $X, Y$ with joint PDF $f_{X,Y}(x,y)$ it holds that $f_{Y|X}(y|x) = F_{X,Y}(x,y)/F_{X}(x)$.
\begin{solution}
It's false.
\end{solution}
\end{truefalse}

\begin{truefalse}
Let $X$ be a discrete rv on the numbers $\{a_{i}\}_{i=1}^{\infty}$ with $a_{i} \in \R$.
Claim: the PMF of $X$ can be found by $f_{X}(x) = F_{X}'(x)$ for $x\in \{a_{i}\}_{i=1}^{\infty}$.
\begin{solution}
It's false. The PMF does not have a derivative (in the proper sense) at $a_{i}$.
\end{solution}
\end{truefalse}

\begin{truefalse}
    Let $X \sim \Norm{0,1}$, $Y=X^2$. Claim: The change of variables formula tells us that:
    $$f_Y(y)=f_X(x)\left|\frac{dx}{dy}\right|=\phi(x)\left|\frac{1}{2x}\right|=\phi(\sqrt{y})\frac{1}{2\sqrt{y}},\quad y>0.$$
    \begin{solution}
        False. $Y=X^2$ is not a strictly increasing (or decreasing) function, so the change of variables formula cannot be applied.
    \end{solution}
\end{truefalse}

\begin{truefalse}
Let $X \sim \Gamm{n,\lambda}$, then $\E{X^{k}} = \frac{n+k-1}{\lambda} \E{X^{k-1}}$ for $k \in \N = \{0, 1, 2, \ldots\}$.
Claim, for $c\in \N$,
\begin{equation}
\E{X^{c}} =  \frac{(n+c-1)!}{(n-1)!\lambda^{c}}.
\end{equation}
    \begin{solution}
        TRUE,
        \\This expression is equivalent to
        \\$$\E{X^{c}} = \frac{\Gamma(n+c)}{\Gamma(n)} = \frac{(n+c-1)!}{(n-1)!}$$
        \\This is defined as the c-th moment for Gamma(n, 1). (See Blitzstein page 390)


    \end{solution}
\end{truefalse}

%Alternate version
\begin{truefalse}
Let $X \sim \Gamm{n,\lambda}$, then the c-th moment is given with the expression below. Here n is an integer.
\\$$\E{X^{c}} =  \frac{(n+c-1)!}{(n-1)!}$$
    \begin{solution}
        FALSE,
        \\It should be:
        \\$$\E{X^{c}} = \frac{1}{\lambda^{c}}\frac{\Gamma(n+c)}{\Gamma(n)} = \frac{1}{\lambda^{c}}\frac{(n+c-1)!}{(n-1)!}$$
        \\This is defined as the c-th moment for Gamma(n, $\lambda$). (See Blitzstein page 390 )


    \end{solution}
\end{truefalse}

\begin{truefalse}
  Let $X$ and $Y$ be independent discrete rvs, $T = X + Y$. Claim:
  $$F_T(t) = \sum_x F_X(x - t) p_X(x)$$
\begin{solution}
  False, it should be $F_Y(t - x)$.
  Variations:
  \begin{itemize}
  \item True
  \item False with $p_X(t)$
  \item False with $\sum_x p_Y(t -  x) p_X(x)$
  \item False with $\sum_{x=0}^t F_Y(t - x) p_X(x)$
  \end{itemize}
\end{solution}
\end{truefalse}

\end{document}
