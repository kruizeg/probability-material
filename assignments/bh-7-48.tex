\subsection{BH.7.48}

Read this exercise first and solve it. Then consider the code below.

\begin{exercise}
Include your hand written solution for this problem. Read the instructions on why we want you to do it like this.
\end{exercise}


\begin{minted}[]{python}
import numpy as np

np.random.seed(3)


def find_number_of_maxima(X):
    num_max = 0
    M = -np.infty
    for x in X:
        if x > M:
            num_max += 1
            M = x
    return num_max


num = 10
X = np.random.uniform(size=num)
print(X)

print(find_number_of_maxima(X))

samples = 100
Y = np.zeros(samples)
for i in range(samples):
    X = np.random.uniform(size=num)
    Y[i] = find_number_of_maxima(X)

print(Y.mean(), Y.var(), Y.std())
\end{minted}


\begin{minted}[]{R}
set.seed(3)

find_number_of_maxima = function(X) {
  num_max = 0
  M = -Inf
  for (x in X) {
    if(x > M) {
      num_max = num_max + 1
      M = x
    }
  }
  return(num_max)
}


num = 10
X = runif(num, min = 0, max = 1)
print(X)

print(find_number_of_maxima(X))

samples = 100
Y = rep(0, samples)
for (i in 1:samples) {
  X = runif(num, min = 0, max = 1)
  Y[i] = find_number_of_maxima(X)
}

print(mean(Y))
print(var(Y))
print(sd(Y))
\end{minted}

\begin{exercise}
Explain how the small function in lines P.6 to P.13 (R.4-R.12) works.
(You should know that \texttt{x += 1} is an extremely useful abbreviation of the code \texttt{x = x + 1}. Unfortunately, R does not have such increment operators.)
\end{exercise}

\begin{exercise}
Explain the code in lines P.25 and P.26 (R.25, R.26).
\end{exercise}
